\section{Discussion}

\subsection{Glycan Assignment Performance}

    \subsubsection{IgG}
        In \igg and \agp the known common glycoforms were assigned without ambiguity,
    both with and without formate adducts, confirmed manually. In these cases,
    network smoothing is not necessary. The estimates of $\mathbf{\tau}$ in \igg
    are low because the number of glycan compositions observed is low and the overlap
    in $\mathbf{A}$ is large.

    \subsubsection{Alpha-1 Acid Glycoprotein}
        In \agp, there are many more compositions to assign, which in turn leads to
    larger $\mathbf{\tau}$ estimates, with good support for all observations with the
    exception of \texttt{\{Hex:5; HexNAc:4; Neu5Ac:1\}} which depends upon the
    asialo-bi-antennary neighborhoods, and is also the only observed member of this
    neighborhoods. The related sample \dpagp, shows similar distributions for tri-,
    tetra, and penta-antennary forms with low abundance of the bi-antennary forms. The
    low abundance mono-sialylated tetra-antennary case is 


    \subsubsection{Influenza Strains}
        In \phil, the unregularized case contains some ambiguous matches where the biological
    context implies they should not be possible, \texttt{\{Hex:7; HexNAc:6; Neu5Ac:4\}}, or
    where the biological context and neighboring observations support the presence of a glycan
    composition but the evidence does not satisfy the scoring function, \texttt{\{Hex:10; HexNAc:2\}},
    \texttt{\{Hex:6; HexNAc:5\}}, and \texttt{\{Fuc:1; Hex:7; HexNAc:6\}}. By applying the smoothing
    procedure with the parameters automatically estimated with grid search (Table \ref{tbl:phil_82_score_table}),
    \texttt{\{Hex:7; HexNAc:6; Neu5Ac:4\}} was eliminated, while \texttt{\{Hex:10; HexNAc:2\}} and
    \texttt{\{Hex:6; HexNAc:5\}} were boosted into a higher confidence score range. The change
    to \texttt{\{Fuc:1; Hex:7; HexNAc:6\}} was insufficient to reach a high confidence score range,
    and the score for \texttt{\{Fuc:1; Hex:8; HexNAc:7\}} was dropped from a plausible score range
    to a low confidence range. Both these larger asialo-\nglycans have been previously assigned
    in \cite{Khatri2016a},  but the automated procedure drops these compositions while estimating
    ${\hat \gamma}$ leading to empty neighborhoodss when estimating $\mathbf{\tau}$. A user-generated
    $\mathbf{\tau}$ would contain a value greater than 5 in those neighborhoods.
    \todo[inline]{This would be another table entry in \verb|tbl:phil82_score_table|, already computed but
    uncertain how to integrate into the flow yet}

        We are justified in \texttt{\{Hex:7; HexNAc:6; Neu5Ac:4\}}'s removal based on its lack of supporting
    intermediary glycoforms and relatives above the selected $\hat{\gamma}$. If its own score exceeded $\hat{\gamma}$,
    it would itself result in a non-zero value for its related neighborhoods, providing itself with a
    non-zero minimum value and adjusting its rate of decay with $\lambda$. This is not to say that the
    LC-MS evidence that was observed is not real signal, merely that the assignment of that signal the
    glycan composition \texttt{\{Hex:7; HexNAc:6; Neu5Ac:4\}} is unlikely given the context.

        The related \dpphil sample we see a similar pattern of glycoforms though with a wider range of
    fucosylation. In this case, we also have chemical noise from permethylation and a low degree of
    ammonium adduction, which can lead to more spurious matches. We match several multiply
    sialylated glycan compositions in this sample which do not satisfy any of the neighborhood conditions
    in Table \ref{tbl:neighborhood_definitions}, which results in their score decaying rapidly as $\lambda\rightarrow1$.
    If these compositions were viable under the user's glycome model, then a different set of neighborhood
    rules would need to be specified which covered this group. In this case, several of them are ambiguous
    assignments of the same signal due to the mass shift imposed by ammonium ($17.026$ Da, \texttt{H3 N1}) compared to a proton
    adduct, is nearly the same as the difference between a permethylated \monosaccharide{Neu5Ac} and permethylated
    \monosaccharide{FucHex} ($17.015$ Da, \texttt{H3 C1 O1 N-1}). In other cases, monosialylated compositions
    are either not eliminated or receive a larger score after smoothing because they are connected to the
    asialo neighborhoods, which are strongly supported in this glycome. We manually confirmed that the
    signal assigned to \texttt{\{{Fuc:1; Hex:5; HexNAc:4; Neu5NAc:1}\}} is not a deconvolution artefact,
    but the signal for \texttt{\{{Fuc:2; Hex:5; HexNAc:4; Neu5NAc:1}\}} is partially overlapped and 
    difficult to manually separate. While this connection between monosialo and asialo forms may make sense
    in some cases, the link may not be appropriate for the biological context of this sample where
    a viral protein Neuraminadase removes \monosaccharide{Neu5Ac}, and the user could redefine the
    neighborhoods to omit that overlap. Similar commentary can be made for \philbs, which while not
    permethylated and ammoniated, shows the presence of sialylated compositions due to variable adduction.

    \subsection{Utility of Network Smoothing}
