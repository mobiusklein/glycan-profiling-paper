\subsection{Feature Evaluation}
    We computed several metrics to estimate
    how distinguishable each candidate feature was from random noise. The
    metrics are mentioned in List~\ref{list:scoring_features}, but for more
    information see Section~S\ref{S-sec:feature_evaluation}.
    \begin{ordlist}
    \begin{enumerate}
        \itemsep0em
        \caption{Chromatographic Feature Metrics\label{list:scoring_features}}
        \item Goodness-of-fit of chromatographic peak shape to a model function
              (\cite{Yu2010,Kronewitter2014}).
        \item Goodness-of-fit of isotopic pattern to glycan composition weighted
              by peak abundance (\cite{Maxwell2012}).
        \item Observed charge states with respect to glycan composition and mass.
        \item Time gap between $MS^1$ observations detecting missing peaks
              and interference.
        \item Adduction states with respect to glycan composition and mass.
    \end{enumerate}
    \end{ordlist}

    These metrics are bounded in $[0, 1)$. Any observation for which any metric
    was observed below $0.15$ was discarded as having insufficient evidence for
    consideration. The observed score $s$ for each candidate feature is
    the sum of the logit-transformation of these metrics. This produces a single
    value bounded in $[0, \infty)$, whose distribution we assume is asymptotically
    normal. A value of $s < 8$ reflects a low confidence match, with confidence increasing
    as $s$ does. As these metrics are tied to reliable detection of the the glycan
    by the mass spectrometer, they are dependent upon glycan abundance and sample
    quality and the resolution of the mass spectrometer used.
