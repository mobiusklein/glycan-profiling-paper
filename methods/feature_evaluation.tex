\subsection{Feature Evaluation}
    We computed several metrics to estimate
    how distinguishable each candidate feature was from random noise. The
    metrics are mentioned in List~\ref{list:scoring_features}, but for more
    information see Section~S\sref{S-sec:feature_evaluation}.
    \begin{ordlist}
    \begin{enumerate}
        \itemsep0em
        \item Goodness-of-fit of chromatographic peak shape to a model function
              (\cite{Yu2010,Kronewitter2014}).
        \item Goodness-of-fit of isotopic pattern to glycan composition weighted
              by peak abundance (\cite{Maxwell2012}).
        \item Observed charge states with respect to glycan composition and mass.
        \item Time gap between $MS^1$ observations detecting missing peaks
              and interference.
        \item Adduction states with respect to glycan composition and mass.
        \caption{Chromatographic Feature Metrics\label{list:scoring_features}}
    \end{enumerate}
    \end{ordlist}

    These metrics are bounded in $(-\infty, 1)$. Any observation for which any metric
    was observed below a feature specific threshold was discarded as having insufficient
    evidence for consideration. The observed score $s$ for each candidate feature is
    the sum of the logit-transformation of these metrics. This produces a single
    value bounded in $(-\infty, \infty)$, whose distribution we assume is asymptotically
    normal. A value of $s < 8$ reflects a low confidence match, with confidence increasing
    as $s$ does. As these metrics are tied to reliable detection of the the glycan
    by the mass spectrometer, they depend upon glycan abundance, sample quality and
    mass spectrometer resolution.
