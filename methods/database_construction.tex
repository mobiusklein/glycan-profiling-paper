\subsection{Glycan Hypothesis Generation}
    In eukaryotes, \nglycans start with a common, conserved core of \textbf{HexNAc2 Hex3},
    building up to \textbf{HexNAc2 Hex9} (\cite{Stanley2009}). This structure is refined by
    sequentially removing monosaccharides and replacing them with more complex structures
    through a series of glycosylase and glycosyltransferase reactions, the enumeration of,
    which as shown in \cite{Akune2016}, yields over a million of possible \nglycan topologies.
    These topologies define the geometry of the glycan, affecting the glycan's binding affinities
    and how the glycan may influence protein folding and accessibility, the glycan's functional
    aspects. The medium through which we observed glycans did not capture the full tree or
    graph structure of an \nglycan, so we reduced the topology to a count of each type of residue.

    Starting with the core motif, we generated all combinations of monosaccharides ranging
    between the limits in Table~\ref{tab:glycan_composition_rules} to build a glycan composition
    database, which produced 1240 distinct compositions. We created a copy of this database for
    native, reduced and permethylated, and deuteroreduced and permethylated for each experimental
    protocol we analyzed in this study. We chose to use a combinatorial database for simplicity.
    The later algorithms can be used with an arbitrary glycan composition list. This places the
    burden of finding or creating such a list on the user. The glycan database is stored in a
    SQLite3 (v3.15.2) database file (\cite{Hipp2016}).

    \begin{table}
        \small
        \centering
        \caption{Glycan Composition Rule Table}\label{tab:glycan_composition_rules}
        \begin{threeparttable}
        \begin{tabular}{c | c | c | c}
            \toprule
            Monosaccharide & Lower Limit & Upper Limit & Constraints\\
            \midrule
            \monosaccharide{HexNAc} & 2 & 9 &\\
            \monosaccharide{Hex} & 3 & 10 & \\
            \monosaccharide{Fuc} & 0 & 4 & $\monosaccharide{HexNAc} > \monosaccharide{Fuc}$\\
            \monosaccharide{NeuAc} & 0 & 5 & $(\monosaccharide{HexNAc} - 1) > \monosaccharide{NeuAc}$\\
        \end{tabular}
        \end{threeparttable}
    \end{table}
