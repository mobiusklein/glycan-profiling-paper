\subsection{Glycan Hypothesis Generation}

    In eukaryotes, a 14 monosaccharide \nglycan of composition \textbf{HexNAc2 Hex12} is
    transferred to a newly synthesized protein in the endoplasmic reticulum by
    the oligosaccharyl transferase protein complex.  This glycan is trimmed to
    \textbf{HexNAc2 Hex9} during protein folding and quality control.  As the glycoprotein
    transits the Golgi apparatus, \nglycans are trimmed to \textbf{HexNAc2 Hex5} before
    being elaborated into hybrid and complex \nglycan classes (\citealp{Stanley2009}).
    Glycan structures are refined by a series of reactions that yield over a million
    possible \nglycan topologies, as shown in \citealp{Akune2016}. These topologies define
    the glycan's geometry and protein binding properties. Neither $MS^1$ nor collisional
    tandem $MS$ of glycans can capture the full tree or graph structure of an \nglycan,
    so we reduced the topology to a count of each type of residue, a composition.

    Starting with the core motif \textbf{HexNAc2 Hex3}, we generated all combinations of
    monosaccharides ranging between the limits in Table~\ref{tab:glycan_composition_rules}
    to build a human \nglycan composition database, which produced 1240 distinct compositions.
    These rules are able to efficiently generate all glycan compositions from canonical branching
    patterns and lactosamine extensions, as well as rarer constructs such as LacdiNAc \citealp{Goldberg2009}
    at the cost of including some wholly improbable compositions.
    To perform a side-by-side comparison we also extracted the glycan list from \citealp{Yu2013}
    derived from the biosynthetic rules in \citealp{Krambeck2005} with 319 compositions, and
    another database using all curated \nglycans from \glyspace via GlyTouCan (\citealp{Tiemeyer2017})
    containing only \textbf{[Hex, HexNAc, Fuc, Neu5Ac, sulfate]}, with 275 distinct compositions. As
    previous analysis of Influenza A virus samples detected sulfated \nglycans (\citealp{Khatri2016a}),
    we also created a combinatorial database with up to one sulfate included, for a total
    of 2480 compositions. As our algorithm treats \textbf{HexNAc} and \textbf{HexNAc(S)}
    as distinct entities, for all monosaccharides with post-attachment substituents such
    as \textbf{sulfate} and \textbf{phosphate}, we detached the substituent from the core
    monosaccharide. Our implementation is able to interpret IUPAC trivial names and compositions
    thereof with standard substituent and unambiguous backbone modifications, permitting a
    wide range of possible glycan compositions.

    \begin{table}[tb]
        \scriptsize
        \centering
        \begin{threeparttable}
        \begin{tabular}{c | c | c | c}
            \toprule
            Monosaccharide & Lower Limit & Upper Limit & Constraints\\
            \midrule
            \monosaccharide{HexNAc} & 2 & 9 &\\
            \monosaccharide{Hex} & 3 & 10 & \\
            \monosaccharide{Fuc} & 0 & 4 & $\monosaccharide{HexNAc} > \monosaccharide{Fuc}$\\
            \monosaccharide{NeuAc} & 0 & 5 & $(\monosaccharide{HexNAc} - 1) > \monosaccharide{NeuAc}$\\
        \end{tabular}
        \end{threeparttable}
        \caption{Human \nglycan Composition Bounds \citealp{Stanley2009}}\label{tab:glycan_composition_rules}
    \end{table}
