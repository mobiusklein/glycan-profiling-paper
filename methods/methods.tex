\section{Methods}

\subsection{Glycan Hypothesis Generation}
        In eukaryotes, \nglycans start with a common, conserved core of \textbf{HexNAc2 Hex3},
    building up to \textbf{HexNAc2 Hex9} (\cite{Stanley2009}). This structure is refined by
    sequentially removing monosaccharides and replacing them with more complex structures
    through a series of glycosidase and glycosyltransferase reactions, the enumeration of
    which as shown in \cite{Akune2016} yields over a million of possible \nglycan topologies
    and epitopes. These topologies define the geometry of the glycan, affecting the glycan's
    binding affinities and how the glycan may influence protein folding and accessibility,
    the glycan's functional aspects. The medium through which we observe \nglycan does not
    capture the full tree or graph structure of an \nglycan, so we reduce the topology to
    a count of each type of residue.

        Starting with the core motif, we generate all combinations of monosaccharides ranging
    between the limits in Table \ref{tab:glycan_composition_rules}. We created a copy of
    this database for native, reduced and permethylated, and deuteroreduced and permethylated.
    Let $n = 1240$ be the number of glycan compositions $\mathbf{g}$ in the database.

    \renewcommand{\arraystretch}{1.5}
    \begin{table}
        \centering
        \savenotes
        \caption{Glycan Composition Rule Table}\label{tab:glycan_composition_rules}
        \begin{tabular}{c | c | c | c}
            Monosaccharide & Lower Limit & Upper Limit & Constraints\\
            \hline
            \monosaccharide{HexNAc} & 2 & 9 &\\
            \monosaccharide{Hex} & 3 & 10 & \\
            \monosaccharide{Fuc} & 0 & 4 & $\monosaccharide{HexNAc} > \monosaccharide{Fuc}$\\
            \monosaccharide{NeuAc} & 0 & 5 & $(\monosaccharide{HexNAc} - 1) > \monosaccharide{NeuAc}$\\
        \end{tabular}
        \spewnotes
    \end{table}
    \renewcommand{\arraystretch}{1.0}

\subsection{LC-MS Data Preprocessing}
    We analyzed samples from several sources, including both QTOF and Orbitrap
    instruments as shown in Table \ref{tab:sample_overview}. For details on sample preparation
    and data acquisition, please see the source citations. We converted all
    datasets to mzML format (\cite{Martens2011}) using Proteowizard
    (\cite{Kessner2008}) without any data transforming filters. We applied a background reduction
    method based upon (\cite{Kaur2006}), using a window length of 2 m/z. Next, we picked peaks using
    a simple Gaussian model and iteratively charge state deconvoluted and deisotoped using an
    averagine (\cite{Senko1995}) formula appropriate to the molecule under study. For native
    glycans, the formula was \textbf{H 1.690 C 1.0 O 0.738 N 0.071}, for permethylated glycans,
    the formula was \textbf{H 1.819 C 1.0 O 0.431 N 0.042}. We used an iterative approach which combines
    aspects of the dependence graph method (\cite{Liu2010}) and with subtraction. All samples
    were processed using a minimum isotopic fit score of 20 with an isotopic strictness penalty
    of 2.

    \begin{table}
        \caption{Samples Used}\label{tab:sample_overview}
        \small
        \centering
        \begin{threeparttable}
        \begin{tabular}{p{4.1cm} | c | p{3cm} | p{3cm} | c}
            \toprule
            Sample Name & Instrument & Derivatization & Adduction & Source\\
            \midrule
            20150930-06-AGP & QTOF & Native & Formate (1) & \cite{Khatri2016a}\\
            20141031-07-Phil-82 & QTOF & Native & Formate (3) & \cite{Khatri2016a}\\
            20141103-02-Phil-BS & QTOF & Native & Formate (3) & \cite{Khatri2016a}\\
            20151002-02-IGG & QTOF & Native & Formate (2) & \cite{Khatri2016b}\\
            20141128-11-Phil-82\tnote{1} & QTOF &
                Deutero-reduced and Permethylated & Ammonium (3) & \cite{Khatri2016a}\\
            AGP-DR-Perm-glycans-1\tnote{1} & Orbitrap &
                Deutero-reduced and Permethylated & Ammonium (3) & \cite{Khatri2016a}\\
            AGP-permethylated-2ul-inj-55-SLens\tnote{1} & Orbitrap &
                Reduced and Permethylated & Ammonium (3) & \cite{Khatri2016a}\\
            Perm-BS-070111-04-Serum\tnote{1} & Orbitrap &
                Reduced and Permethylated & Ammonium (3) & \cite{Yu2013,Hu2012}\\
        \end{tabular}
        \begin{tablenotes}
            \item[1] Included $MS^n$ Scans
        \end{tablenotes}
        \end{threeparttable}
    \end{table}


\subsection{Chromatogram Aggregation}
        We cluster peaks whose neutral masses are within 15 parts-per-million error
    (PPM) of each other. When there are multiple candidate clusters for a single peak,
    we use the cluster with the lowest mass error. After all peaks are clustered,
    we sort each cluster by time, creating a list of aggregated chromatograms. To account
    for small mass differences, we find all chromatograms which are within 10 PPM of each
    other and which overlap in time and merge them.

\subsection{Glycan Composition Matching}
        For each chromatogram, we queried the target glycan database for compositions
    whose masses were within $\delta_{mass} = 10$ PPM for QTOF data, $5$ PPM
    for FTMS data. We merged all features matching the same composition. Then, for
    each adduct combination, we searched the target glycan database for compositions
    whose neutral mass were within $\delta_{mass}$ of the observed neutral mass - adduct
    combination mass, followed by another round of merging chromatogramss with the same
    assigned composition. We reduced the data by splitting each feature where the time
    between sequential observation was greater than $\delta_{rt} = 0.25$ minutes and
    removed features with fewer than $k = 5$ data points. For cases where $MS^n$ scans were
    present, these scans were mapped onto their precursor $MS^1$ features, and evaluated for
    glycan-like product ions, retaining only those which satisfied the signature ion
    criterion described in section S\ref{S-sec:signature_ion_criterion}. We term the
    remaining assigned and unassigned chromatograms \textit{candidate features}.


\subsection{Feature Evaluation}
        For each candidate feature, we computed several metrics to estimate
    how distinguishable the observed signal was from random noise. The
    features are mentioned in List~\ref{list:scoring_features}, but for more
    information see Section~S\ref{S-sec:feature_evaluation}.
    \begin{ordlist}
    \begin{enumerate}
        \itemsep0em
        \caption{Chromatographic Feature Metrics\label{list:scoring_features}}
        \item Goodness-of-fit of chromatographic peak shape to a model function
              (\cite{Yu2010,Kronewitter2014}).
        \item Goodness-of-fit of isotopic pattern to glycan composition weighted
              by peak abundance (\cite{Maxwell2012}).
        \item Observed charge states with respect to glycan composition and mass.
        \item Time gap between $MS^1$ observations detecting measuring missing peaks
              and interference.
        \item Adduction states with respect to glycan composition and mass.
    \end{enumerate}
    \end{ordlist}

        These metrics are bounded in $[0, 1)$. Any observation for which any metric
    was observed below $0.15$ was discarded as having insufficient evidence for
    consideration. The \textit{observed score} $s$ for each candidate feature is
    the sum of the logit-transformation of these metrics. This produces a single
    value bounded in $[0, \infty)$, whose distribution we assume is asymptotically
    normal. $s < 8$ reflects a low confidence match, with confidence increasing
    as $s$ does. As these metrics are tied to reliable detection of the the glycan
    by the mass spectrometer, they are dependent upon glycan abundance and sample
    quality and the resolution of the mass spectrometer used.


\subsection{Glycan Composition Network Smoothing}
        Evidence for individual glycan compostions can often be enough to claim
    that composition had been detected. Lower abundance may score poorly in one
    or more features, leading to the glycan composition being discarded. Other
    methods have demonstrated it is advantageous to use relationships between
    glycans based on biosynthetic or structural rules to adjust the score of a
    single glycan assignment (\cite{Goldberg2009, Kronewitter2014}). This idea
    has been explored more generically under the name "Manifold Regularization"
    (\cite{Belkin2006}) and specifically "Laplacian Regularization" when the
    Laplacian matrix of a graph is used to influence the parameter scaling. We
    apply this idea to weighted networks of related glycans with arbitrarily
    defined and overlapping sub-populations.

        \subsubsection{Glycan Composition Graph}
        For each database of theoretical glycan compositions we create, we
        define each composition to be a coordinate vector in a $\mathcal{Z}^{+c}$
        space where $c$ is the number of components in any glycan composition,
        and represented by a node in an undirected glycan composition
        graph $\mathcal{G}$. Under this interpretation, we can compute the
        $L_1$-distance between two glycan compositions. For any two glycan
        compositions $g_u, g_v$, if $L_1(g_u, g_v) = 1$ we add an edge
        connecting $g_u$ and $g_v$ to $\mathcal{G}$ with weight $w = 1$.


        \subsubsection{Neighborhood Definition}
        Our definition of distance connects glycan compositions which differ
        by a single monosaccharide, but we can assert how larger collections of
        glycan compositions are related. To this end, we extend the definition of
        neighborhoods for \nglycans using intervals over monosaccharide counts
        shown in Table~\ref{tab:neighborhood_definitions}. These neighborhoods are
        arranged to span particular epitopes or biosynthetically related
        subtypes of \nglycans, such as sialylation state or branching
        pattern.

        \begin{table}[tb]
            \centering
            \small
            \begin{tabular}[h]{l p{8cm}}
                \toprule
                Name & Bounds \\
                \midrule
                High Mannose & $\monosaccharide{HexNAc} = 2 \land
                                \monosaccharide{Hex} \in [3, 10]
                                \land \monosaccharide{NeuAc} = 0$\\
                Hybrid & $\monosaccharide{HexNAc} \in [2, 4] \land
                          \monosaccharide{Hex} \in [2, 6]
                          \land \monosaccharide{NeuAc} \in [0, 2]$\\
                Bi-Antennary & $\monosaccharide{HexNAc} \in [3, 5]
                                \land \monosaccharide{Hex} \in [3, 6]
                                \land \monosaccharide{NeuAc} \in [1, 3]$\\
                Asialo-Bi-Antennary & $\monosaccharide{HexNAc} \in [3, 5]
                                \land \monosaccharide{Hex} \in [3, 6]
                                \land \monosaccharide{NeuAc} \in [0, 1]$\\
                Tri-Antennary & $
                    \monosaccharide{HexNAc} \in [4, 6]
                    \land \monosaccharide{Hex} \in [4, 7]
                    \land \monosaccharide{NeuAc} \in [1, 4]
                $\\
                Asialo-Tri-Antennary & $
                    \monosaccharide{HexNAc} \in [4, 6]
                    \land \monosaccharide{Hex} \in [4, 7]
                    \land \monosaccharide{NeuAc} \in [0, 0]
                $\\
                Tetra-Antennary & $
                    \monosaccharide{HexNAc} \in [5, 7]
                    \land \monosaccharide{Hex} \in [5, 8]
                    \land \monosaccharide{NeuAc} \in [1, 5]
                $\\
                Asialo-Tetra-Antennary & $
                    \monosaccharide{HexNAc} \in [5, 7]
                    \land \monosaccharide{Hex} \in [5, 8]
                    \land \monosaccharide{NeuAc} \in [0, 0]
                $\\
                Penta-Antennary & $
                    \monosaccharide{HexNAc} \in [6, 8]
                    \land \monosaccharide{Hex} \in [6, 9]
                    \land \monosaccharide{NeuAc} \in [1, 5]
                $\\
                Asialo-Penta-Antennary & $
                    \monosaccharide{HexNAc} \in [6, 8]
                    \land \monosaccharide{Hex} \in [6, 9]
                    \land \monosaccharide{NeuAc} \in [0, 0]
                $\\
                Hexa-Antennary & $
                    \monosaccharide{HexNAc} \in [7, 9]
                    \land \monosaccharide{Hex} \in [7, 10]
                    \land \monosaccharide{NeuAc} \in [1, 6]
                $\\
                Asialo-Hexa-Antennary & $
                    \monosaccharide{HexNAc} \in [7, 9]
                    \land \monosaccharide{Hex} \in [7, 10]
                    \land \monosaccharide{NeuAc} \in [0, 0]
                $\\
                Hepta-Antennary & $
                    \monosaccharide{HexNAc} \in [8, 10]
                    \land \monosaccharide{Hex} \in [8, 11]
                    \land \monosaccharide{NeuAc} \in [1, 7]
                $\\
                Asialo-Hepta-Antennary & $
                    \monosaccharide{HexNAc} \in [8, 10]
                    \land \monosaccharide{Hex} \in [8, 11]
                    \land \monosaccharide{NeuAc} \in [0, 0]
                $
            \end{tabular}
            \caption{N-Glycan Neighborhoods}
            \label{tab:neighborhood_definitions}
        \end{table}

        Glycan compositions may belong to zero or more neighborhoods,
        as there are unusual glycan compositions which do not satisfy
        any neighborhood's rules, and several neighborhoods intentionally
        overlap to express broad relationships between groups.

        We define a matrix $\mathbf{A}$ as an $n \times k$ matrix where
        $A_{i, k}$ to be the degree to which $g_i$ belongs $k$th neighborhood:

        \begin{align}
            A_{i, k} &= \frac{1}{|\text{neighborhood}_k|}\sum_{
                g^* \in \text{neighborhood}_k}{L_1(g_i, g^*)}
        \end{align}

        % The stated reduction is not well tested, and the change
        % may well be minimal because all that really happens is
        % the weight of the column for each row is weighted by a
        % shrinking function of column size. It may be better if
        % we don't manipulate A at all.

        To reduce the impact of neighborhood size on the elements
        of $\mathbf{A}$, the columns of $\mathbf{A}$ are first
        normalized to sum to 1, and then the rows of $\mathbf{A}$
        are normalized to sum to 1\reviewfootnote{
            The stated reduction is not well tested, and the change
            may well be minimal because all that really happens is
            the weight of the column for each row is weighted by a
            shrinking function of column size. It may be better if
            we don't manipulate A at all.
        }.

        We assume that members of the same neighborhood will
        share a central tendency, $\mathbf{\tau}$.


        \subsubsection{Laplacian Regularization}
        % Lacking citations for the fundamentals of this material
        We combine the observed score $\mathbf{s}$ and the structure
        of $\mathcal{G}$ to estimate a smoothed score $\mathbf{\phi}$
        that combines the evidence for each individual glycan composition
        as well as its relatives. As $\mathbf{s}$ is the size of the
        set of observed glycan composition $p$ while $\mathbf{\phi}$
        is of size $n$, we partition $\mathbf{\phi}$ into a block
        vector $\begin{bmatrix}\phi_o\\ \phi_m\end{bmatrix}$ with
        dimensions $\begin{bmatrix}p\\ n-p\end{bmatrix}$.

        Let $\mathbf{L}$ be the weighted Laplacian matrix of $\mathcal{G}$,
        which is an $n \times n$ matrix. To ensure $\mathbf{L}$ is
        invertible, we add $\mathbf{I}_n$ to $\mathbf{L}$. We partition
        $\mathbf{L}$ into blocks $\begin{bmatrix} \mathbf{L_{oo}} &
        \mathbf{L_{om}} \\ \mathbf{L_{mo}} & \mathbf{L_{mm}}\end{bmatrix}$.
        We also partition $\mathbf{A}$ into $\begin{bmatrix}\mathbf{A}_o\\
        \mathbf{A}_m\end{bmatrix}$ and $\tau_o = \mathbf{A}_o\tau$,
        $\tau_m = \mathbf{A}_m\tau$.

        We find the $\mathbf{\phi}$ that minimizes the expression
        \begin{align}
            \ell &= (\mathbf{s} - \mathbf{\phi_o})^t(\mathbf{s} - \mathbf{\phi_o}) + \lambda
                \begin{bmatrix}
                    \phi_o - \tau_o, & \phi_m - \tau_m
                \end{bmatrix}
                \begin{bmatrix}
                    \mathbf{L_{oo}} & \mathbf{L_{om}} \\ \mathbf{L_{mo}} & \mathbf{L_{mm}}
                \end{bmatrix}
                \begin{bmatrix}
                    \phi_o - \tau_o \\ \phi_m - \tau_m
                \end{bmatrix} \label{eqn:laplacian_regularization_objective_function}
        \end{align}
        \noindent where $\lambda$ controls how much weight is
        placed on the network structure and $\tau$.

        To obtain the optimal $\mathbf{\phi}$, we take the partial
        derivative of $\ell$ w.r.t $\phi_m$

        \begin{align}
            0 &= \frac{\partial\ell}{\partial\phi_m}\left((\mathbf{s} - \mathbf{\phi_o})^t(\mathbf{s} - \mathbf{\phi_o}) + \lambda
                \begin{bmatrix}
                    \phi_o - \tau_o, & \phi_m - \tau_m
                \end{bmatrix}
                \begin{bmatrix}
                    \mathbf{L_{oo}} & \mathbf{L_{om}} \\ \mathbf{L_{mo}} & \mathbf{L_{mm}}
                \end{bmatrix}
                \begin{bmatrix}
                    \phi_o - \tau_o \\ \phi_m - \tau_m
                \end{bmatrix}\right)\\
            % (\phi_m - \tau_m) &= -\mathbf{L_{mm}}^{-1}\mathbf{L_{mo}}(\phi_o - \tau_o) \nonumber\\
            {\hat \phi_m} &= -\mathbf{L_{mm}}^{-1}\mathbf{L_{mo}}(\phi_o - \tau_o) + \tau_m
            \label{eqn:estimate_of_phi_m}
        \end{align}

        \noindent and w.r.t. $\phi_o$

        \begin{align}
            0 &= \frac{\partial\ell}{\partial\phi_o}\left((\mathbf{s} - \mathbf{\phi_o}
                )^t(\mathbf{s} - \mathbf{\phi_o}) + \lambda
                \begin{bmatrix}
                    \phi_o - \tau_o, & \phi_m - \tau_m
                \end{bmatrix}
                \begin{bmatrix}
                    \mathbf{L_{oo}} & \mathbf{L_{om}} \\ \mathbf{L_{mo}} & \mathbf{L_{mm}}
                \end{bmatrix}
                \begin{bmatrix}
                    \phi_o - \tau_o \\ \phi_m - \tau_m
                \end{bmatrix}\right)\\
            {\hat \phi_o} &= \left[
                \mathbf{I} + \lambda\left(\mathbf{L_{oo}} -
                    \mathbf{L_{om}}\mathbf{L_{mm}^{-1}}\mathbf{L_{mo}}
                \right)
            \right]^{-1}(\mathbf{s} - \tau_o) + \tau_o
            \label{eqn:estimate_of_phi_o}
        \end{align}

        To use this method, we must provide values for $\lambda$ and $\mathbf{\tau}$.
        While these values could be chosen based on the expectations of the user for
        a given experiment, we provide an algorithm for selecting their values.
        These methods use the topology of the glycan composition graph and the
        distribution of observed scores, and cannot fully capture boundary cases
        or related but disconnected parts of the graph.


    \subsubsection{Parameter Estimation}
    We model the relationship between $\mathbf{s}$, $\mathbf{\phi_o}$, and
    $\mathbf{\tau}$ as a set of Gaussian distribution.
    \begin{align}
        \left(\mathbf{s}|\mathbf{\phi_o}, \mathbf{\tau}\right) &\sim
            \mathcal{N}(\mathbf{\phi_o}, \Sigma)\\
        \Sigma &= \rho\mathbf{I}
    \end{align}
    \begin{align}
        \left(\begin{bmatrix}
            \mathbf{\phi_o}\\
            \mathbf{\phi_m}
        \end{bmatrix}\middle|\mathbf{\tau}\right) &\sim
            \mathcal{N}(\mathbf{A\tau}, \lambda^{-1}\mathbf{L}^-)\\
        \left(\mathbf{\phi_o}\middle|\mathbf{\tau}\right) &\sim
            \mathcal{N}\left(\mathbf{A_o}\mathbf{\tau}, \Sigma_{\phi_o}\right)\\
        \Sigma_{\phi_o} &= \lambda^{-1}\left(
            \mathbf{L_{oo}} - \mathbf{L_{om}L_{mm}^{-1}L_{mo}}\right)^{-1}\\
        \mathbf{\tau} &\sim \mathcal{N}\left(0, \sigma^2\mathbf{I}\right)
    \end{align}

    \noindent Fully expanded, this becomes
    \begin{equation}
        \begin{bmatrix}
            \mathbf{s}\\
            \mathbf{\phi_o}\\
            \mathbf{\tau}
        \end{bmatrix} \sim \mathcal{N}\left(
            \begin{bmatrix}0\\0\\0\end{bmatrix},
            \begin{bmatrix}
                \Sigma + \Sigma_{\phi_o} + \sigma^2\mathbf{A_oA_o}^t &
                \Sigma_{\phi_o} + \sigma^2\mathbf{A_oA_o}^t &
                \sigma^2\mathbf{A_o}\\
                \Sigma_{\phi_o} + \sigma^2\mathbf{A_oA_o}^t &
                \Sigma_{\phi_o} + \sigma^2\mathbf{A_oA_o}^t &
                \sigma^2\mathbf{A_o}\\
                \sigma^2\mathbf{A_o}^t & \sigma^2\mathbf{A_o}^t & \sigma^2\mathbf{I}\\
            \end{bmatrix}
        \right)\label{eqn:multivariate_gaussian_model}
    \end{equation}

    We can form the conditional distribution $\tau|\mathbf{s}$ which has a mean

    \begin{align}
        \mu_{\tau|\mathbf{s}} &= 0 + (\sigma^2\mathbf{A_o}^t)\left(
            \Sigma + \Sigma_{\phi_o} + \sigma^2\mathbf{A_oA_o^t}\right)^{-1}\mathbf{s}\\
        % &= \mathbf{A_o}^t\left(
        %     \frac{\rho}{\sigma^2}\mathbf{I} + \frac{1}{\lambda\sigma^2}\mathbf{L_{oo}^-} + 
        %     \mathbf{A_oA_o^t}
        %     \right)^{-1}\mathbf{s} \nonumber\\
        &= \mathbf{A_o}^t\left(
            {\tilde\rho}\mathbf{I} + \frac{1}{{\tilde\lambda}}\mathbf{L_{oo}^-} + 
            \mathbf{A_oA_o^t}
            \right)^{-1}\mathbf{s} \label{eqn:tau_given_s}
    \end{align}

        We assume that $\sigma^2 \gg 1$, and treat $\lambda$ and $\rho$
    as relative to $\sigma^2$, as ${\tilde \rho}$ and ${\tilde \lambda}$.
    This model gives us an estimate for $\tau$ given a value for
    $\rho$ and $\lambda$. As $\rho$ has no direct role in the central
    tendency of $\mathbf{\phi}$ or $\mathbf{s}$, we choose to fix the
    value of ${\tilde \rho} = 0.1$, which leaves only ${\tilde \lambda}$.
    We estimate the optimal ${\tilde \lambda}$ by grid search, minimizing
    the predicted residual error sum of squares (PRESS) statistic.

    \begin{align}
        \argmin_{\tilde \lambda} & \frac{\mathbf{s - {\hat \phi_o}}}{\left(
            1 - \left(
                \mathbf{I} + {\tilde \lambda}\mathbf{L}
            \right)^{-1}
        \right)^2}
    \end{align}

        This formulation depends upon the value of \textbf{s} and is
    sensitive to low scoring matches, which can lead to incorrect
    estimates of $\tau$ and PRESS. We therefore perform a grid
    search over both ${\tilde \lambda}$ and a minimum threshold
    for \textbf{s}, $\gamma$.

    % Does this network pruning merit a pseudo-code section?
        As we increase $\gamma$ we remodel the graph $\mathcal{G}$,
    removing nodes whose score is below $\gamma$. For each pair
    of neighbors of removed node $g_m$, $(g_u, g_v)$, if
    $L_1(g_u, g_v) >  L_1(g_u, g_m) + L_1(g_m, g_v)$, we add an
    edge from $g_u$ to $g_v$ with weight $\frac{1}{L_1(g_u, g_m)
    + L_1(g_m, g_v)}$, up to a limit of $L_1(g_k, g_m) < 5$.
    We give the result of this grid search the name $\mathbf{r}$.
    At each point, on the grid, we save the value of $\tau$ in
    $r_{\lambda_i, \gamma_j, \tau}$ and the PRESS in $r_{
    \lambda_i, \gamma_j, PRESS}$. To select the optimal parameters,
    we traverse the grid along $\gamma$, computing $\mathbf{\tau_\gamma}$:

    \begin{align}
        {\bar \lambda_j} &= \argmin_{\lambda_i}{r_{\lambda_i, \gamma_j, PRESS}} \\
        \tau_{\gamma_j} &= |r_{{\bar \lambda_j}, \gamma_j, \tau}| * \left(
            \frac{\gamma_j}{b} + (1 - \frac{1}{b})\right)
    \end{align}

    \noindent where $b$ is a bias factor defining how much
    weight to give to higher values of $\gamma$ which
    correspond to networks made up of higher confidence
    assignments. We chose $b = 4$. We define ${\bar \tau_\gamma} =
    \max{\mathbf{\tau_\gamma}}$ and define the  vector
    $\mathbf{\bar \gamma} = \left[\gamma_j \leftarrow\tau_{\gamma_j}
    \ge {\bar \tau_\gamma} * 0.9\right]$. This favors values of
    $\gamma$ where large values of $\tau$ are selected, meaning that
    the neighborhoods are well populated, while also giving an estimate
    for ${\tilde \lambda}$ that is non-zero. We term the values of
    $\gamma$ in $\mathbf{{\bar \gamma}}$ the {\em target thresholds}
    of \textbf{s}.

        To estimate ${\tilde \lambda}$ and $\tau$ from these results,
    we select the columns of the grid $\mathbf{r}$ at each $\gamma_j
    \in \mathbf{{\bar \gamma}}$ and applied the following procedure:

    \begin{align}
    % The maximum tau from the grid search over gamma
    {\bar \tau_\gamma} &= \max{\mathbf{\tau_\gamma}}\\
    % Those values of gamma whose tau is within 10% of the maximum value of tau
    % observed
    \mathbf{\bar \gamma} &= \left\{\gamma_j \leftarrow\tau_{\gamma_j}
        \ge {\bar \tau_\gamma} * 0.9\right\}\\
    % The PRESS minimizing lambda values assocaited with these selected gamma
    {\bar \lambda} &= \left\{ {\bar \lambda_j} \leftarrow \gamma_j \in {\bar \gamma}\right\}\\
    % The observed scores in the partitions which exceed the threshold gamma
    \mathbf{s_{\gamma_j}} &= \left\{s_i \leftarrow s_i > \gamma_j\right\} \\
    % The new estimated tau based upon the selected partion
    \mathbf{{\bar \tau_j}} &= \mu_{\tau|\mathbf{s}_{\gamma_j}, {\bar \lambda}_j}\\
    % The average selected lambda
    {\hat \lambda} &= \frac{1}{|\mathbf{{\bar \lambda}}|}\sum_j {\bar \lambda}_j\\
    % The average selected tau
    {\hat \tau} &= \frac{1}{|\mathbf{{\bar \tau}}|}\sum_j \mathbf{{\bar \tau_j}}\\
    % The average threshold gamma
    {\hat \gamma} &= \frac{1}{|\mathbf{{\bar \gamma}}|}\sum_j {\bar \gamma}_j
    \end{align}

    \noindent where $\mathbf{s}_{\gamma_j}$ is the set of observed
    scores which are greater than $\gamma_j$, but where the estimation
    of is carried out with the complete Laplacian $\mathbf{L}$,
    not the reduced network used to compute $\mathbf{r}$. This set of
    averaged estimates of ${\hat \lambda}$ and ${\hat \tau}$ are then
    used to estimate ${\hat \phi_o}$ by \eqref{eqn:estimate_of_phi_o}.

\subsection{Performance Comparison}
    We compare the performance of the described algorithm with and without
    network smoothing. State of the art glycan LC-MS profiling software
    has been designed around Thermo-Fisher Scientific instrumentation, with
    support for their binary format (\cite{Kronewitter2014}, \cite{Yu2013})
    but not open community formats. MultiGlycan-ESI, though publicly available,
    was unable to be applied to the majority of our datasets because they were
    not acquired on that vendor's instruments, and their mzXML alternative
    did not produce matches consistent with their previously published results
    on \textit{Perm-BS-070111-04-Human-Serum}, and when ran on
    \textit{AGP-permethylated-2ul-inj-55-SLens} it ran out of memory. GlyQ-IQ
    was made available for testing by its authors, but required Thermo-Fisher
    binary format, but, as it assumed that glycans were in native form, and it
    did not make its test data publicly available.
