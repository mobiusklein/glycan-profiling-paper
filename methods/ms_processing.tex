\subsection{LC-MS Data Preprocessing}
    We analyzed samples from several sources, including both Quadrupole Time-of-Flight (QTOF) and
    Orbitrap instruments as shown in Table S\sref{S-tab:sample_overview}. For details on sample preparation
    and data acquisition, please see the source citations in the referenced table. We converted all
    datasets to mzML format (\citealp{Martens2011}) using Proteowizard (\citealp{Kessner2008}) without
    any data transforming filters. We applied a background reduction method based upon (\citealp{Kaur2006}),
    using a window length of 2 m/z. Next, we picked peaks using a Gaussian model and iteratively charge
    state deconvoluted and deisotoped using an averagine (\citealp{Senko1995}) formula appropriate to the
    molecule under study. For native glycans, the formula was \textbf{H 1.690 C 1.0 O 0.738 N 0.071},
    for permethylated glycans, the formula was \textbf{H 1.819 C 1.0 O 0.431 N 0.042}. We used an iterative
    approach which combines aspects of the dependence graph method (\citealp{Liu2010}) and with subtraction.
    All samples were processed using a minimum isotopic fit score of 20 with an isotopic strictness penalty
    of 2.

