    \subsubsection{Neighborhood Definition}
        Our definition of distance connects glycan compositions which differ
        by a single monosaccharide, but we can assert how larger collections of
        glycan compositions are related. To this end, we extend the definition of
        neighborhoods for \nglycans using intervals over monosaccharide counts
        shown in Table~\ref{tab:neighborhood_definitions}. These neighborhoods are
        arranged to span particular epitopes or biosynthetically related
        subtypes of \nglycans, such as sialylation state or branching
        pattern. Neighborhoods overlap sets of glycan compositions which are also
        biosynthetically related. Each neighborhood spans the eponymous class of
        glycan compositions, as well as the preceding class and proceeding class.
        For example, the Tri-Antennary neighborhood spans Bi-Antennary and
        Tetra-Antennary compositions. This helps to channel the estimation of
        $\mathbf{\tau}$ among related groups.
        The Hybrid, Bi-Antennary and Asialo-Bi-Antennary neighborhoods introduce
        complications because they are biosynthetically close to each other. For the
        simplicity, we chose to include all of Hybrid in Asialo-Bi-Antennary
        and permit up to one NeuAc  \reviewchange{in its members.}

        \begin{table}[tb]
            \scriptsize
            \begin{tabular}{l |S|S| S|S| S|S| S}
                \toprule
                \multirow{3}{*}{Name} & \multicolumn{2}{c|}{HexNAC} & \multicolumn{2}{c|}{Hex} & \multicolumn{2}{c|}{NeuAc} & {Size} \\
                                      &  {Min}        &    {Max}   &     {Min}   &    {Max}  &     {Min}     &    {Max}  &      \\
                \midrule
                High Mannose          &     2         &     2      &       3     &     10    &       0       &      0    &  16 \\
                Hybrid                &     2         &     4      &       2     &     6     &       0       &      2    &  80 \\
                Bi-Antennary          &     3         &     5      &       3     &     6     &       1       &      3    &  104\\
                Asialo-Bi-Antennary   &     3         &     5      &       3     &     6     &       0       &      1    &  96 \\
                Tri-Antennary         &     4         &     6      &       4     &     7     &       1       &      4    &  172\\
                Asialo-Tri-Antennary  &     4         &     6      &       4     &     7     &       0       &      0    &  56 \\
                Tetra-Antennary       &     5         &     7      &       5     &     8     &       1       &      5    &  240\\
                Asialo-Tetra-Antennary&     5         &     7      &       5     &     8     &       0       &      0    &  60\\
                Penta-Antennary       &     6         &     8      &       6     &     9     &       1       &      5    &  280\\
                Asialo-Penta-Antennary&     6         &     8      &       6     &     9     &       0       &      0    &  60\\
                Hexa-Antennary        &     7         &     9      &       7     &     10    &       1       &      6    &  300\\
                Asialo-Hexa-Antennary &     7         &     9      &       7     &     10    &       0       &      0    &  60\\
                Hepta-Antennary       &     8         &     10     &       8     &     11    &       1       &      7    &  150\\
                Asialo-Hepta-Antennary&     8         &     10     &       8     &     11    &       0       &      0    &  30\\
                \bottomrule
            \end{tabular}
            \caption{N-Glycan Neighborhood Definitions. These define the ranges of monosaccharides which
            will be used to classify a glycan composition as being a member of each neighborhood, and the number
            of \textit{combinatorial} \nglycan compositions in each neighborhood.}
            \label{tab:neighborhood_definitions}
        \end{table}

        Glycan compositions may belong to zero or more neighborhoods,
        as there are unusual glycan compositions which do not satisfy
        any neighborhood's rules, and several neighborhoods intentionally
        overlap to express broad relationships between groups.

        We define a matrix $\mathbf{A}$ as an $n \times k$ matrix where
        $A_{i, k}$ is the degree to which $g_i$ belongs $k$th neighborhood:
        \begin{align}
            A_{i, k} &= \frac{1}{|\text{neighborhood}_k|}\sum_{
                g^* \in \text{neighborhood}_k}{L_1(g_i, g^*)}
        \end{align}
        To reduce the impact of neighborhood size on the elements
        of $\mathbf{A}$, the columns of $\mathbf{A}$ are first
        normalized to sum to 1, and then the rows of $\mathbf{A}$
        are normalized to sum to 1. We assume that members of the
        same neighborhood will share a central tendency $\mathbf{\tau}$.
