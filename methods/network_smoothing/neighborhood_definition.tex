    \subsubsection{Neighborhood Definition}
        Our definition of distance connects glycan compositions which differ
        by a single monosaccharide, but we can assert larger collections of
        glycan compositions are related. We define the following neighborhoods
        for {\em N}-glycans:

        \begin{table}
            \centering
            \begin{tabular}[h]{l | p{8cm}}
                Name & Bounds \\
                \hline
                High Mannose & $\monosaccharide{HexNAc} = 2 \land
                                \monosaccharide{Hex} \in [3, 10]
                                \land \monosaccharide{NeuAc} = 0$\\
                Hybrid & $\monosaccharide{HexNAc} \in [2, 4] \land
                          \monosaccharide{Hex} \in [2, 6]
                          \land \monosaccharide{NeuAc} \in [0, 2]$\\
                Bi-Antennary & $\monosaccharide{HexNAc} \in [3, 5]
                                \land \monosaccharide{Hex} \in [3, 6]
                                \land \monosaccharide{NeuAc} \in [1, 3]$\\
                Asialo-Bi-Antennary & $\monosaccharide{HexNAc} \in [3, 5]
                                \land \monosaccharide{Hex} \in [3, 6]
                                \land \monosaccharide{NeuAc} \in [0, 1]$\\
                Tri-Antennary & $
                    \monosaccharide{HexNAc} \in [4, 6]
                    \land \monosaccharide{Hex} \in [4, 7]
                    \land \monosaccharide{NeuAc} \in [1, 4]
                $\\
                Asialo-Tri-Antennary & $
                    \monosaccharide{HexNAc} \in [4, 6]
                    \land \monosaccharide{Hex} \in [4, 7]
                    \land \monosaccharide{NeuAc} \in [0, 0]
                $\\
                Tetra-Antennary & $
                    \monosaccharide{HexNAc} \in [5, 7]
                    \land \monosaccharide{Hex} \in [5, 8]
                    \land \monosaccharide{NeuAc} \in [1, 5]
                $\\
                Asialo-Tetra-Antennary & $
                    \monosaccharide{HexNAc} \in [5, 7]
                    \land \monosaccharide{Hex} \in [5, 8]
                    \land \monosaccharide{NeuAc} \in [0, 0]
                $\\
                Penta-Antennary & $
                    \monosaccharide{HexNAc} \in [6, 8]
                    \land \monosaccharide{Hex} \in [6, 9]
                    \land \monosaccharide{NeuAc} \in [1, 5]
                $\\
                Asialo-Penta-Antennary & $
                    \monosaccharide{HexNAc} \in [6, 8]
                    \land \monosaccharide{Hex} \in [6, 9]
                    \land \monosaccharide{NeuAc} \in [0, 0]
                $
            \end{tabular}
            \caption{N-Glycan Neighborhoods}
            \label{tbl:neighborhood_definitions}
        \end{table}

        Glycan compositions may belong to zero or more neighborhoods,
        as there are unusual glycan compositions which do not satisfy
        any neighborhood's rules, and several neighborhoods intentionally
        overlap to express broad relationships between groups. We define
        a matrix $\mathbf{A}$ as an $n \times k$ matrix where $A_{i, k}$
        to be the degree to which $g_i$ belongs $k$th neighborhood:

        \begin{align}
            A_{i, k} &= \frac{1}{|\text{neighborhood}_k|}\sum_{
                g^* \in \text{neighborhood}_k}{L_1(g_i, g^*)}
        \end{align}

        To reduce the impact of neighborhood size on the elements
        of $\mathbf{A}$, the columns of $\mathbf{A}$ are first
        normalized to sum to 1, and then the rows of $\mathbf{A}$
        are normalized to sum to 1.

        We assume that members of the same neighborhood will
        share a central tendency, $\mathbf{\tau}$.
