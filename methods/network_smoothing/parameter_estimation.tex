\subsubsection{Parameter Estimation}
        We model the relationship between $\mathbf{s}$, $\mathbf{\phi_o}$, and
    $\mathbf{\tau}$ as a set of gaussian distribution.
    \begin{align}
        \left(\mathbf{s}|\mathbf{\phi_o}, \mathbf{\tau}\right) &\sim
            \mathcal{N}(\mathbf{\phi_o}, \Sigma)\\
        \Sigma &= \rho\mathbf{I}
    \end{align}
    \begin{align}
        \left(\begin{bmatrix}
            \mathbf{\phi_o}\\
            \mathbf{\phi_m}
        \end{bmatrix}\middle|\mathbf{\tau}\right) &\sim
            \mathcal{N}(\mathbf{A\tau}, \lambda^{-1}\mathbf{L}^-)\\
        \left(\mathbf{\phi_o}\middle|\mathbf{\tau}\right) &\sim
            \mathcal{N}\left(\mathbf{A_o}\mathbf{\tau}, \Sigma_{\phi_o}\right)\\
        \Sigma_{\phi_o} &= \lambda^{-1}\left(
            \mathbf{L_{oo}} - \mathbf{L_{om}L_{mm}^{-1}L_{mo}}\right)^{-1}\\
        \mathbf{\tau} &\sim \mathcal{N}\left(0, \sigma^2\mathbf{I}\right)
    \end{align}

    \noindent Fully expanded, this becomes
    \begin{equation}
        \begin{bmatrix}
            \mathbf{s}\\
            \mathbf{\phi_o}\\
            \mathbf{\tau}
        \end{bmatrix} \sim \mathcal{N}\left(
            \begin{bmatrix}0\\0\\0\end{bmatrix},
            \begin{bmatrix}
                \Sigma + \Sigma_{\phi_o} + \sigma^2\mathbf{A_oA_o}^t &
                \Sigma_{\phi_o} + \sigma^2\mathbf{A_oA_o}^t &
                \sigma^2\mathbf{A_o}\\
                \Sigma_{\phi_o} + \sigma^2\mathbf{A_oA_o}^t &
                \Sigma_{\phi_o} + \sigma^2\mathbf{A_oA_o}^t &
                \sigma^2\mathbf{A_o}\\
                \sigma^2\mathbf{A_o}^t & \sigma^2\mathbf{A_o}^t & \sigma^2\mathbf{I}\\
            \end{bmatrix}
        \right)\label{eqn:multivariate_gaussian_model}
    \end{equation}

    We can form the conditional distribution $\tau|\mathbf{s}$ which has a mean

    \begin{align}
        \mu_{\tau|\mathbf{s}} &= 0 + (\sigma^2\mathbf{A_o}^t)\left(
            \Sigma + \Sigma_{\phi_o} + \sigma^2\mathbf{A_oA_o^t}\right)^{-1}\mathbf{s}\\
        % &= \mathbf{A_o}^t\left(
        %     \frac{\rho}{\sigma^2}\mathbf{I} + \frac{1}{\lambda\sigma^2}\mathbf{L_{oo}^-} + 
        %     \mathbf{A_oA_o^t}
        %     \right)^{-1}\mathbf{s} \nonumber\\
        &= \mathbf{A_o}^t\left(
            {\tilde\rho}\mathbf{I} + \frac{1}{{\tilde\lambda}}\mathbf{L_{oo}^-} + 
            \mathbf{A_oA_o^t}
            \right)^{-1}\mathbf{s} \label{eqn:tau_given_s}
    \end{align}

        We assume that $\sigma^2 \gg 1$, and treat $\lambda$ and $\rho$
    as relative to $\sigma^2$, as ${\tilde \rho}$ and ${\tilde \lambda}$.
    This model gives us an estimate for $\tau$ given a value for
    $\rho$ and $\lambda$. As $\rho$ has no direct role in the central
    tendency of $\mathbf{\phi}$ or $\mathbf{s}$, we choose to fix the
    value of ${\tilde \rho} = 0.1$, which leaves only ${\tilde \lambda}$.
    We estimate the optimal ${\tilde \lambda}$ by grid search, minimizing
    the predicted residual error sum of squares (PRESS) statistic.

    \begin{align}
        \argmin_{\tilde \lambda} & \frac{\mathbf{s - {\hat \phi_o}}}{\left(
            1 - \left(
                \mathbf{I} + {\tilde \lambda}\mathbf{L}
            \right)^{-1}
        \right)^2}
        % \\
        % Since within this procedure, we are dealing with not the full
        % Laplacian matrix but the Laplacian matrix of the reduced network
        % should the notation for \mathbf{L}'s blocks be altered to reflect
        % this?
        % 
        % This full expansion of the expression is not necessary in the main text
        % \argmin_{\tilde \lambda} & \frac{\mathbf{s} - \left(\left[
        %     \mathbf{I} + {\tilde \lambda}\left(\mathbf{L_{oo}} -
        %         \mathbf{L_{om}}\mathbf{L_{mm}^{-1}}\mathbf{L_{mo}}
        %     \right)\right]^{-1}(\mathbf{s} - \tau_o) + \tau_o\right)}{\left(
        %     1 - \left(
        %         \mathbf{I} + {\tilde \lambda}\mathbf{L}
        %     \right)^{-1}
        % \right)^2} \nonumber\\
        % \argmin_{\tilde \lambda} & \frac{
        %     \mathbf{s} - \left(\left[
        %         \mathbf{I} + {\tilde \lambda}\left(\mathbf{L_{oo}} -
        %             \mathbf{L_{om}}\mathbf{L_{mm}^{-1}}\mathbf{L_{mo}}
        %         \right)\right]^{-1}\left(\mathbf{s} - \mathbf{A_o}^t\left(
        %         {\tilde\rho}\mathbf{I} + \frac{1}{{\tilde\lambda}}\mathbf{L_{oo}^-} + 
        %         \mathbf{A_oA_o^t}
        %         \right)^{-1}\mathbf{s}\right) + \mathbf{A_o}^t\left(
        %         {\tilde\rho}\mathbf{I} + \frac{1}{{\tilde\lambda}}\mathbf{L_{oo}^-} + 
        %         \mathbf{A_oA_o^t}
        %         \right)^{-1}\mathbf{s}\right)}{
        %         \left(
        %             1 - \left(
        %                 \mathbf{I} + {\tilde \lambda}\mathbf{L}
        %             \right)^{-1}
        %         \right)^2} \label{eqn:press_for_lambda}
    \end{align}

        This formulation depends upon the value of \textbf{s} and is
    sensitive to low scoring matches, which can lead to incorrect
    estimates of $\tau$ and PRESS. We therefore perform a grid
    search over both ${\tilde \lambda}$ and a minimum threshold
    for \textbf{s}, $\gamma$.

    % Does this network pruning merit a pseudo-code section?
        As we increase $\gamma$ we remodel the graph $\mathcal{G}$,
    removing nodes whose score is below $\gamma$. For each pair
    of neighbors of removed node $g_m$, $(g_u, g_v)$, if
    $L_1(g_u, g_v) >  L_1(g_u, g_m) + L_1(g_m, g_v)$, we add an
    edge from $g_u$ to $g_v$ with weight $\frac{1}{L_1(g_u, g_m)
    + L_1(g_m, g_v)}$, up to a limit of $L_1(g_k, g_m) < 5$.
    We give the result of this grid search the name $\mathbf{r}$.
    At each point, on the grid, we save the value of $\tau$ in
    $r_{\lambda_i, \gamma_j, \tau}$ and the PRESS in $r_{
    \lambda_i, \gamma_j, PRESS}$. To select the optimal parameters,
    we traverse the grid along $\gamma$, computing $\mathbf{\tau_\gamma}$:

    \begin{align}
        {\bar \lambda_j} &= \argmin_{\lambda_i}{r_{\lambda_i, \gamma_j, PRESS}} \\
        \tau_{\gamma_j} &= |r_{{\bar \lambda_j}, \gamma_j, \tau}| * \left(
            \frac{\gamma_j}{b} + (1 - \frac{1}{b})\right)
    \end{align}

    \noindent where $b$ is a bias factor defining how much
    weight to give to higher values of $\gamma$ which
    correspond to networks made up of higher confidence
    assignments. We chose $b = 4$. We define ${\bar \tau_\gamma} =
    \max{\mathbf{\tau_\gamma}}$ and define the  vector
    $\mathbf{\bar \gamma} = \left[\gamma_j \leftarrow\tau_{\gamma_j}
    \ge {\bar \tau_\gamma} * 0.9\right]$. This favors values of
    $\gamma$ where large values of $\tau$ are selected, meaning that
    the neighborhoods are well populated, while also giving an estimate
    for ${\tilde \lambda}$ that is non-zero. We term the values of
    $\gamma$ in $\mathbf{{\bar \gamma}}$ the {\em target thresholds}
    of \textbf{s}.

        To estimate ${\tilde \lambda}$ and $\tau$ from these results,
    we select the columns of the grid $\mathbf{r}$ at each $\gamma_j
    \in \mathbf{{\bar \gamma}}$.

    \begin{align}
    % The maximum tau from the grid search over gamma
    {\bar \tau_\gamma} &= \max{\mathbf{\tau_\gamma}}\\
    % Those values of gamma whose tau is within 10% of the maximum value of tau
    % observed
    \mathbf{\bar \gamma} &= \left\{\gamma_j \leftarrow\tau_{\gamma_j}
        \ge {\bar \tau_\gamma} * 0.9\right\}\\
    % The PRESS minimizing lambda values assocaited with these selected gamma
    {\bar \lambda} &= \left\{ {\bar \lambda_j} \leftarrow \gamma_j \in {\bar \gamma}\right\}\\
    % The observed scores in the partitions which exceed the threshold gamma
    \mathbf{s_{\gamma_j}} &= \left\{s_i \leftarrow s_i > \gamma_j\right\} \\
    % The new estimated tau based upon the selected partion
    \mathbf{{\bar \tau_j}} &= \mu_{\tau|\mathbf{s}_{\gamma_j}, {\bar \lambda}_j}\\
    % The average selected lambda
    {\hat \lambda} &= \frac{1}{|\mathbf{{\bar \lambda}}|}\sum_j {\bar \lambda}_j\\
    % The average selected tau
    {\hat \tau} &= \frac{1}{|\mathbf{{\bar \tau}}|}\sum_j \mathbf{{\bar \tau_j}}\\
    % The average threshold gamma
    {\hat \gamma} &= \frac{1}{|\mathbf{{\bar \gamma}}|}\sum_j {\bar \gamma}_j
    \end{align}

    \noindent where $\mathbf{s}_{\gamma_j}$ is the set of observed
    scores which are greater than $\gamma_j$, but where the estimation
    of is carried out with the complete Laplacian $\mathbf{L}$,
    not the reduced network used to compute $\mathbf{r}$. This set of
    averaged estimates of ${\hat \lambda}$ and ${\hat \tau}$ are then
    used to estimate ${\hat \phi_o}$ by \eqref{eqn:estimate_of_phi_o}.