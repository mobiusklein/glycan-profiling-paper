    \subsubsection{Laplacian Regularization}
        % Lacking citations for the fundamentals of this material
        We combine the observed score $\mathbf{s}$ and the structure
        of $\mathcal{G}$ to estimate a smoothed score $\mathbf{\phi}$
        that combines the evidence for each individual glycan composition
        as well as its relatives. As $\mathbf{s}$ is the size of the
        set of observed glycan composition $p$ while $\mathbf{\phi}$
        is of size $n$, we partition $\mathbf{\phi}$ into a block
        vector $\begin{bmatrix}\phi_o\\ \phi_m\end{bmatrix}$ with
        dimensions $\begin{bmatrix}p\\ n-p\end{bmatrix}$.

        Let $\mathbf{L}$ be the weighted Laplacian matrix of $\mathcal{G}$,
        which is an $n \times n$ matrix. To ensure $\mathbf{L}$ is
        invertible, we add $\mathbf{I}_n$ to $\mathbf{L}$. We partition
        $\mathbf{L}$ into blocks $\begin{bmatrix} \mathbf{L_{oo}} &
        \mathbf{L_{om}} \\ \mathbf{L_{mo}} & \mathbf{L_{mm}}\end{bmatrix}$.
        We also partition $\mathbf{A}$ into $\begin{bmatrix}\mathbf{A}_o\\
        \mathbf{A}_m\end{bmatrix}$ and $\tau_o = \mathbf{A}_o\tau$,
        $\tau_m = \mathbf{A}_m\tau$.

        We find the $\mathbf{\phi}$ that minimizes the expression
        \begin{align}
            \mathcal{S}(\mathbf{L}, \phi, \tau) &= \begin{bmatrix}
                    \phi_o - \tau_o, & \phi_m - \tau_m
                \end{bmatrix}
                \begin{bmatrix}
                    \mathbf{L_{oo}} & \mathbf{L_{om}} \\ \mathbf{L_{mo}} & \mathbf{L_{mm}}
                \end{bmatrix}
                \begin{bmatrix}
                    \phi_o - \tau_o \\ \phi_m - \tau_m
                \end{bmatrix} \\
            \ell &= (\mathbf{s} - \mathbf{\phi_o})^t(\mathbf{s} - \mathbf{\phi_o}) + \lambda
                \mathcal{S}(\mathbf{L}, \phi, \tau) \label{eqn:laplacian_regularization_objective_function}
        \end{align}
        \noindent where $\lambda$ controls how much weight is
        placed on the network structure and $\tau$.

        To obtain the optimal $\mathbf{\phi}$, we take the partial
        derivative of $\ell$ w.r.t $\phi_m$

        \begin{align}
            0 &= \frac{\partial\ell}{\partial\phi_m}\left((\mathbf{s} - \mathbf{\phi_o})^t(\mathbf{s} - \mathbf{\phi_o}) + \lambda
                \mathcal{S}(\mathbf{L}, \phi, \tau)\right)\\
            % (\phi_m - \tau_m) &= -\mathbf{L_{mm}}^{-1}\mathbf{L_{mo}}(\phi_o - \tau_o) \nonumber\\
            {\hat \phi_m} &= -\mathbf{L_{mm}}^{-1}\mathbf{L_{mo}}(\phi_o - \tau_o) + \tau_m
            \label{eqn:estimate_of_phi_m}
        \end{align}

        \noindent and w.r.t. $\phi_o$

        \begin{align}
            0 &= \frac{\partial\ell}{\partial\phi_o}\left((\mathbf{s} - \mathbf{\phi_o}
                )^t(\mathbf{s} - \mathbf{\phi_o}) + \lambda
                \mathcal{S}(\mathbf{L}, \phi, \tau)\right)\\
            {\hat \phi_o} &= \left[
                \mathbf{I} + \lambda\left(\mathbf{L_{oo}} -
                    \mathbf{L_{om}}\mathbf{L_{mm}^{-1}}\mathbf{L_{mo}}
                \right)
            \right]^{-1}(\mathbf{s} - \tau_o) + \tau_o
            \label{eqn:estimate_of_phi_o}
        \end{align}

        To use this method, we must provide values for $\lambda$ and $\mathbf{\tau}$.
        While these values could be chosen based on the expectations of the user for
        a given experiment, we provide an algorithm for selecting their values in Section
        S~\ref{S-sec:laplacian_regularization_parameter_estimation}.
        These methods use the topology of the glycan composition graph and the
        distribution of observed scores, and cannot fully capture boundary cases
        or related but disconnected parts of the graph.
