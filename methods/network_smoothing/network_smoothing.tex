\subsection{Glycan Composition Network Smoothing}

    Ideally, each glycan present in a sample under analysis would produce sufficient
    experimental evidence that they can be identified. In practice, glycan
    compositions with lower abundances may not present strong evidence, leading
    to those glycan compositions being discarded. Others have demonstrated that
    it is advantageous to use relationships between glycans based on biosynthetic
    or structural rules to adjust the score of a single glycan assignment
    (\cite{Goldberg2009, Kronewitter2014}). To improve performance, we propose
    a method based on Laplacian regularized least squares (\cite{Belkin2006})
    to use evidence from glycan compositions related over a network to smooth
    its evaluation of glycan composition feature matching.

    % new material per reviewer request
    \reviewchange{
    Previous approaches to using information regarding identification of one
    glycan composition to increase the confidence in another have been proposed
    by} \cite{Goldberg2009} and \cite{Kronewitter2014} \reviewchange{using different techniques.
    Goldberg et al. used random walks along the biosynthetic network between identified
    glycan compositions to increase the confidence of those connected compositions. This
    method works well but requires that the parameters of the random walk be properly tuned
    for the biosynthetic network being used. Laplacian regularized least squares is more
    robust to small changes to the network and is able to use the entire network. Kronwitter
    et al. included a term in their criterion for detection requiring the presence of another
    glycan composition with one more or one less monosaccharide to permit identification. This
    puts substantial weight on a boolean term, giving it the ability to overrule other
    experimental evidence. Similar methods could be devised using methods like ant colony
    optimization to traverse the biosynthetic graph, or a a database-specific belief network,
    but these methods would require considerable manual tuning for each new database to be
    tested.}

        \subsubsection{Glycan Composition Graph}
        For each database of theoretical glycan compositions we create, we
        define each composition to be a coordinate vector in a $\mathcal{Z}^{+c}$
        space where $c$ is the number of components in any glycan composition,
        and represented by a node in an undirected glycan composition
        graph $\mathcal{G}$. Under this interpretation, we can compute the
        $L_1$-distance between two glycan compositions, \reviewchange{representing the biosynthetic
        distance between the two compositions, an analog for the number of enzymatic
        steps needed to go from one glycan to the other}. For any two glycan
        compositions $g_u, g_v$, if $L_1(g_u, g_v) = 1$ we add an edge
        connecting $g_u$ and $g_v$ to $\mathcal{G}$ with weight $w = 1$.


        \subsubsection{Neighborhood Definition}
        Our definition of distance connects glycan compositions which differ
        by a single monosaccharide, but we can assert how larger collections of
        glycan compositions are related. To this end, we extend the definition of
        neighborhoods for \nglycans using intervals over monosaccharide counts
        shown in Table~\ref{tab:neighborhood_definitions}. These neighborhoods are
        arranged to span particular epitopes or biosynthetically related
        subtypes of \nglycans, such as sialylation state or branching
        pattern. \reviewchange{Neighborhoods overlap sets of glycan compositions which are also
        biosynthetically related. Each neighborhood spans the eponymous class of
        glycan compositions, as well as the preceding class and proceeding class.
        For example, For example, the Tri-Antennary neighborhood spans Bi-Antennary
        Tetra-Antennary compositions. This helps to channel the estimation of}
        $\mathbf{\tau}$ \reviewchange{among related groups.
        The Hybrid, Bi-Antennary and Asialo-Bi-Antennary neighborhoods introduce
        complications because they are biosynthetically close to each other. For the
        simplicity, we chose to include all of Hybrid in Asialo-Bi-Antennary
        and permit up to one NeuAc in it.}

        \begin{table}[tb]
            \scriptsize
            \begin{tabular}{l |S|S| S|S| S|S| S}
                \toprule
                \multirow{3}{*}{Name} & \multicolumn{2}{c|}{HexNAC} & \multicolumn{2}{c|}{Hex} & \multicolumn{2}{c|}{NeuAc} & \reviewchange{Size} \\
                                      &  {Min}        &    {Max}   &     {Min}   &    {Max}  &     {Min}     &    {Max}  &      \\
                \midrule
                High Mannose          &     2         &     2      &       3     &     10    &       0       &      0    &  16 \\
                Hybrid                &     2         &     4      &       2     &     6     &       0       &      2    &  80 \\
                Bi-Antennary          &     3         &     5      &       3     &     6     &       1       &      3    &  104\\
                Asialo-Bi-Antennary   &     3         &     5      &       3     &     6     &       0       &      1    &  96 \\
                Tri-Antennary         &     4         &     6      &       4     &     7     &       1       &      4    &  172\\
                Asialo-Tri-Antennary  &     4         &     6      &       4     &     7     &       0       &      0    &  56 \\
                Tetra-Antennary       &     5         &     7      &       5     &     8     &       1       &      5    &  240\\
                Asialo-Tetra-Antennary&     5         &     7      &       5     &     8     &       0       &      0    &  60\\
                Penta-Antennary       &     6         &     8      &       6     &     9     &       1       &      5    &  280\\
                Asialo-Penta-Antennary&     6         &     8      &       6     &     9     &       0       &      0    &  60\\
                Hexa-Antennary        &     7         &     9      &       7     &     10    &       1       &      6    &  300\\
                Asialo-Hexa-Antennary &     7         &     9      &       7     &     10    &       0       &      0    &  60\\
                Hepta-Antennary       &     8         &     10     &       8     &     11    &       1       &      7    &  150\\
                Asialo-Hepta-Antennary&     8         &     10     &       8     &     11    &       0       &      0    &  30\\
                \bottomrule
            \end{tabular}
            \caption{N-Glycan Neighborhood Definitions. These define the ranges of monosaccharides which
            will be used to classify a glycan composition as being a member of each neighborhood, and the number
            of \textit{Combinatorial} \nglycan compositions in each neighborhood.}
            \label{tab:neighborhood_definitions}
        \end{table}

        Glycan compositions may belong to zero or more neighborhoods,
        as there are unusual glycan compositions which do not satisfy
        any neighborhood's rules, and several neighborhoods intentionally
        overlap to express broad relationships between groups.

        We define a matrix $\mathbf{A}$ as an $n \times k$ matrix where
        $A_{i, k}$ is the degree to which $g_i$ belongs $k$th neighborhood:
        \begin{align}
            A_{i, k} &= \frac{1}{|\text{neighborhood}_k|}\sum_{
                g^* \in \text{neighborhood}_k}{L_1(g_i, g^*)}
        \end{align}
        To reduce the impact of neighborhood size on the elements
        of $\mathbf{A}$, the columns of $\mathbf{A}$ are first
        normalized to sum to 1, and then the rows of $\mathbf{A}$
        are normalized to sum to 1. We assume that members of the
        same neighborhood will share a central tendency $\mathbf{\tau}$.


        \subsubsection{Laplacian Regularization}
        % Lacking citations for the fundamentals of this material
        We combine the observed score $\mathbf{s}$ and the structure
        of $\mathcal{G}$ to estimate a smoothed score $\mathbf{\phi}$
        that combines the evidence for each individual glycan composition
        as well as its relatives. As $\mathbf{s}$ is the size of the
        set of observed glycan composition $p$ while $\mathbf{\phi}$
        is of size $n$, we partition $\mathbf{\phi}$ into a block
        vector $\begin{bmatrix}\phi_o\\ \phi_m\end{bmatrix}$ with
        dimensions $\begin{bmatrix}p\\ n-p\end{bmatrix}$.

        Let $\mathbf{L}$ be the weighted Laplacian matrix of $\mathcal{G}$,
        which is an $n \times n$ matrix. To ensure $\mathbf{L}$ is
        invertible, we add $\mathbf{I}_n$ to $\mathbf{L}$. We partition
        $\mathbf{L}$ into blocks $\begin{bmatrix} \mathbf{L_{oo}} &
        \mathbf{L_{om}} \\ \mathbf{L_{mo}} & \mathbf{L_{mm}}\end{bmatrix}$.
        We also partition $\mathbf{A}$ into $\begin{bmatrix}\mathbf{A}_o\\
        \mathbf{A}_m\end{bmatrix}$ and $\tau_o = \mathbf{A}_o\tau$,
        $\tau_m = \mathbf{A}_m\tau$.

        We find the $\mathbf{\phi}$ that minimizes the expression
        \begin{align}
            \ell &= (\mathbf{s} - \mathbf{\phi_o})^t(\mathbf{s} - \mathbf{\phi_o}) + \lambda
                \begin{bmatrix}
                    \phi_o - \tau_o, & \phi_m - \tau_m
                \end{bmatrix}
                \begin{bmatrix}
                    \mathbf{L_{oo}} & \mathbf{L_{om}} \\ \mathbf{L_{mo}} & \mathbf{L_{mm}}
                \end{bmatrix}
                \begin{bmatrix}
                    \phi_o - \tau_o \\ \phi_m - \tau_m
                \end{bmatrix} \label{eqn:laplacian_regularization_objective_function}
        \end{align}
        \noindent where $\lambda$ controls how much weight is
        placed on the network structure and $\tau$.

        To obtain the optimal $\mathbf{\phi}$, we take the partial
        derivative of $\ell$ w.r.t $\phi_m$

        \begin{align}
            0 &= \frac{\partial\ell}{\partial\phi_m}\left((\mathbf{s} - \mathbf{\phi_o})^t(\mathbf{s} - \mathbf{\phi_o}) + \lambda
                \begin{bmatrix}
                    \phi_o - \tau_o, & \phi_m - \tau_m
                \end{bmatrix}
                \begin{bmatrix}
                    \mathbf{L_{oo}} & \mathbf{L_{om}} \\ \mathbf{L_{mo}} & \mathbf{L_{mm}}
                \end{bmatrix}
                \begin{bmatrix}
                    \phi_o - \tau_o \\ \phi_m - \tau_m
                \end{bmatrix}\right)\\
            (\phi_m - \tau_m) &= -\mathbf{L_{mm}}^{-1}\mathbf{L_{mo}}(\phi_o - \tau_o) \nonumber\\
            {\hat \phi_m} &= -\mathbf{L_{mm}}^{-1}\mathbf{L_{mo}}(\phi_o - \tau_o) + \tau_m
            \label{eqn:estimate_of_phi_m}
        \end{align}

        \noindent and w.r.t. $\phi_o$

        \begin{align}
            0 &= \frac{\partial\ell}{\partial\phi_o}\left((\mathbf{s} - \mathbf{\phi_o}
                )^t(\mathbf{s} - \mathbf{\phi_o}) + \lambda
                \begin{bmatrix}
                    \phi_o - \tau_o, & \phi_m - \tau_m
                \end{bmatrix}
                \begin{bmatrix}
                    \mathbf{L_{oo}} & \mathbf{L_{om}} \\ \mathbf{L_{mo}} & \mathbf{L_{mm}}
                \end{bmatrix}
                \begin{bmatrix}
                    \phi_o - \tau_o \\ \phi_m - \tau_m
                \end{bmatrix}\right)\\
            {\hat \phi_o} &= \left[
                \mathbf{I} + \lambda\left(\mathbf{L_{oo}} -
                    \mathbf{L_{om}}\mathbf{L_{mm}^{-1}}\mathbf{L_{mo}}
                \right)
            \right]^{-1}(\mathbf{s} - \tau_o) + \tau_o
            \label{eqn:estimate_of_phi_o}
        \end{align}

        To use this method, we must provide values for $\lambda$ and $\mathbf{\tau}$.
        While these values could be chosen based on the expectations of the user for
        a given experiment, we provide an algorithm for selecting their values.
        These methods use the topology of the glycan composition graph and the
        distribution of observed scores, and cannot fully capture boundary cases
        or related but disconnected parts of the graph.


    % this section is far too long to include in the main text
    % so it is included in the supplement instead
    % \subsubsection{Parameter Estimation}
        We model the relationship between $\mathbf{s}$, $\mathbf{\phi_o}$, and
    $\mathbf{\tau}$ as a set of gaussian distribution.
    \begin{align}
        \left(\mathbf{s}|\mathbf{\phi_o}, \mathbf{\tau}\right) &\sim
            \mathcal{N}(\mathbf{\phi_o}, \Sigma)\\
        \Sigma &= \rho\mathbf{I}
    \end{align}
    \begin{align}
        \left(\begin{bmatrix}
            \mathbf{\phi_o}\\
            \mathbf{\phi_m}
        \end{bmatrix}\middle|\mathbf{\tau}\right) &\sim
            \mathcal{N}(\mathbf{A\tau}, \lambda^{-1}\mathbf{L}^-)\\
        \left(\mathbf{\phi_o}\middle|\mathbf{\tau}\right) &\sim
            \mathcal{N}\left(\mathbf{A_o}\mathbf{\tau}, \Sigma_{\phi_o}\right)\\
        \Sigma_{\phi_o} &= \lambda^{-1}\left(
            \mathbf{L_{oo}} - \mathbf{L_{om}L_{mm}^{-1}L_{mo}}\right)^{-1}\\
        \mathbf{\tau} &\sim \mathcal{N}\left(0, \sigma^2\mathbf{I}\right)
    \end{align}

    \noindent Fully expanded, this becomes
    \begin{equation}
        \begin{bmatrix}
            \mathbf{s}\\
            \mathbf{\phi_o}\\
            \mathbf{\tau}
        \end{bmatrix} \sim \mathcal{N}\left(
            \begin{bmatrix}0\\0\\0\end{bmatrix},
            \begin{bmatrix}
                \Sigma + \Sigma_{\phi_o} + \sigma^2\mathbf{A_oA_o}^t &
                \Sigma_{\phi_o} + \sigma^2\mathbf{A_oA_o}^t &
                \sigma^2\mathbf{A_o}\\
                \Sigma_{\phi_o} + \sigma^2\mathbf{A_oA_o}^t &
                \Sigma_{\phi_o} + \sigma^2\mathbf{A_oA_o}^t &
                \sigma^2\mathbf{A_o}\\
                \sigma^2\mathbf{A_o}^t & \sigma^2\mathbf{A_o}^t & \sigma^2\mathbf{I}\\
            \end{bmatrix}
        \right)\label{eqn:multivariate_gaussian_model}
    \end{equation}

    We can form the conditional distribution $\tau|\mathbf{s}$ which has a mean

    \begin{align}
        \mu_{\tau|\mathbf{s}} &= 0 + (\sigma^2\mathbf{A_o}^t)\left(
            \Sigma + \Sigma_{\phi_o} + \sigma^2\mathbf{A_oA_o^t}\right)^{-1}\mathbf{s}\\
        % &= \mathbf{A_o}^t\left(
        %     \frac{\rho}{\sigma^2}\mathbf{I} + \frac{1}{\lambda\sigma^2}\mathbf{L_{oo}^-} + 
        %     \mathbf{A_oA_o^t}
        %     \right)^{-1}\mathbf{s} \nonumber\\
        &= \mathbf{A_o}^t\left(
            {\tilde\rho}\mathbf{I} + \frac{1}{{\tilde\lambda}}\mathbf{L_{oo}^-} + 
            \mathbf{A_oA_o^t}
            \right)^{-1}\mathbf{s} \label{eqn:tau_given_s}
    \end{align}

        We assume that $\sigma^2 \gg 1$, and treat $\lambda$ and $\rho$
    as relative to $\sigma^2$, as ${\tilde \rho}$ and ${\tilde \lambda}$.
    This model gives us an estimate for $\tau$ given a value for
    $\rho$ and $\lambda$. As $\rho$ has no direct role in the central
    tendency of $\mathbf{\phi}$ or $\mathbf{s}$, we choose to fix the
    value of ${\tilde \rho} = 0.1$, which leaves only ${\tilde \lambda}$.
    We estimate the optimal ${\tilde \lambda}$ by grid search, minimizing
    the predicted residual error sum of squares (PRESS) statistic.

    \begin{align}
        \argmin_{\tilde \lambda} & \frac{\mathbf{s - {\hat \phi_o}}}{\left(
            1 - \left(
                \mathbf{I} + {\tilde \lambda}\mathbf{L}
            \right)^{-1}
        \right)^2}
        % \\
        % Since within this procedure, we are dealing with not the full
        % Laplacian matrix but the Laplacian matrix of the reduced network
        % should the notation for \mathbf{L}'s blocks be altered to reflect
        % this?
        % 
        % This full expansion of the expression is not necessary in the main text
        % \argmin_{\tilde \lambda} & \frac{\mathbf{s} - \left(\left[
        %     \mathbf{I} + {\tilde \lambda}\left(\mathbf{L_{oo}} -
        %         \mathbf{L_{om}}\mathbf{L_{mm}^{-1}}\mathbf{L_{mo}}
        %     \right)\right]^{-1}(\mathbf{s} - \tau_o) + \tau_o\right)}{\left(
        %     1 - \left(
        %         \mathbf{I} + {\tilde \lambda}\mathbf{L}
        %     \right)^{-1}
        % \right)^2} \nonumber\\
        % \argmin_{\tilde \lambda} & \frac{
        %     \mathbf{s} - \left(\left[
        %         \mathbf{I} + {\tilde \lambda}\left(\mathbf{L_{oo}} -
        %             \mathbf{L_{om}}\mathbf{L_{mm}^{-1}}\mathbf{L_{mo}}
        %         \right)\right]^{-1}\left(\mathbf{s} - \mathbf{A_o}^t\left(
        %         {\tilde\rho}\mathbf{I} + \frac{1}{{\tilde\lambda}}\mathbf{L_{oo}^-} + 
        %         \mathbf{A_oA_o^t}
        %         \right)^{-1}\mathbf{s}\right) + \mathbf{A_o}^t\left(
        %         {\tilde\rho}\mathbf{I} + \frac{1}{{\tilde\lambda}}\mathbf{L_{oo}^-} + 
        %         \mathbf{A_oA_o^t}
        %         \right)^{-1}\mathbf{s}\right)}{
        %         \left(
        %             1 - \left(
        %                 \mathbf{I} + {\tilde \lambda}\mathbf{L}
        %             \right)^{-1}
        %         \right)^2} \label{eqn:press_for_lambda}
    \end{align}

        This formulation depends upon the value of \textbf{s} and is
    sensitive to low scoring matches, which can lead to incorrect
    estimates of $\tau$ and PRESS. We therefore perform a grid
    search over both ${\tilde \lambda}$ and a minimum threshold
    for \textbf{s}, $\gamma$.

    % Does this network pruning merit a pseudo-code section?
        As we increase $\gamma$ we remodel the graph $\mathcal{G}$,
    removing nodes whose score is below $\gamma$. For each pair
    of neighbors of removed node $g_m$, $(g_u, g_v)$, if
    $L_1(g_u, g_v) >  L_1(g_u, g_m) + L_1(g_m, g_v)$, we add an
    edge from $g_u$ to $g_v$ with weight $\frac{1}{L_1(g_u, g_m)
    + L_1(g_m, g_v)}$, up to a limit of $L_1(g_k, g_m) < 5$.
    We give the result of this grid search the name $\mathbf{r}$.
    At each point, on the grid, we save the value of $\tau$ in
    $r_{\lambda_i, \gamma_j, \tau}$ and the PRESS in $r_{
    \lambda_i, \gamma_j, PRESS}$. To select the optimal parameters,
    we traverse the grid along $\gamma$, computing $\mathbf{\tau_\gamma}$:

    \begin{align}
        {\bar \lambda_j} &= \argmin_{\lambda_i}{r_{\lambda_i, \gamma_j, PRESS}} \\
        \tau_{\gamma_j} &= |r_{{\bar \lambda_j}, \gamma_j, \tau}| * \left(
            \frac{\gamma_j}{b} + (1 - \frac{1}{b})\right)
    \end{align}

    \noindent where $b$ is a bias factor defining how much
    weight to give to higher values of $\gamma$ which
    correspond to networks made up of higher confidence
    assignments. We chose $b = 4$. We define ${\bar \tau_\gamma} =
    \max{\mathbf{\tau_\gamma}}$ and define the  vector
    $\mathbf{\bar \gamma} = \left[\gamma_j \leftarrow\tau_{\gamma_j}
    \ge {\bar \tau_\gamma} * 0.9\right]$. This favors values of
    $\gamma$ where large values of $\tau$ are selected, meaning that
    the neighborhoods are well populated, while also giving an estimate
    for ${\tilde \lambda}$ that is non-zero. We term the values of
    $\gamma$ in $\mathbf{{\bar \gamma}}$ the {\em target thresholds}
    of \textbf{s}.

        To estimate ${\tilde \lambda}$ and $\tau$ from these results,
    we select the columns of the grid $\mathbf{r}$ at each $\gamma_j
    \in \mathbf{{\bar \gamma}}$.

    \begin{align}
    % The maximum tau from the grid search over gamma
    {\bar \tau_\gamma} &= \max{\mathbf{\tau_\gamma}}\\
    % Those values of gamma whose tau is within 10% of the maximum value of tau
    % observed
    \mathbf{\bar \gamma} &= \left\{\gamma_j \leftarrow\tau_{\gamma_j}
        \ge {\bar \tau_\gamma} * 0.9\right\}\\
    % The PRESS minimizing lambda values assocaited with these selected gamma
    {\bar \lambda} &= \left\{ {\bar \lambda_j} \leftarrow \gamma_j \in {\bar \gamma}\right\}\\
    % The observed scores in the partitions which exceed the threshold gamma
    \mathbf{s_{\gamma_j}} &= \left\{s_i \leftarrow s_i > \gamma_j\right\} \\
    % The new estimated tau based upon the selected partion
    \mathbf{{\bar \tau_j}} &= \mu_{\tau|\mathbf{s}_{\gamma_j}, {\bar \lambda}_j}\\
    % The average selected lambda
    {\hat \lambda} &= \frac{1}{|\mathbf{{\bar \lambda}}|}\sum_j {\bar \lambda}_j\\
    % The average selected tau
    {\hat \tau} &= \frac{1}{|\mathbf{{\bar \tau}}|}\sum_j \mathbf{{\bar \tau_j}}\\
    % The average threshold gamma
    {\hat \gamma} &= \frac{1}{|\mathbf{{\bar \gamma}}|}\sum_j {\bar \gamma}_j
    \end{align}

    \noindent where $\mathbf{s}_{\gamma_j}$ is the set of observed
    scores which are greater than $\gamma_j$, but where the estimation
    of is carried out with the complete Laplacian $\mathbf{L}$,
    not the reduced network used to compute $\mathbf{r}$. This set of
    averaged estimates of ${\hat \lambda}$ and ${\hat \tau}$ are then
    used to estimate ${\hat \phi_o}$ by \eqref{eqn:estimate_of_phi_o}.
