\subsection{Chromatogram Aggregation}
    We clustered peaks whose neutral masses were within 15 parts-per-million error
    (PPM) of each other. When there were multiple candidate clusters for a single peak,
    we used the cluster with the lowest mass error. Next, we sorted each cluster by time,
    creating a list of aggregated chromatograms. To account for small mass differences,
    we found all chromatograms which are within 10 PPM of each other and which overlap
    in time and merge them.

\subsection{Glycan Composition Matching}
    For each chromatogram, we searched the glycan database for compositions
    whose masses were within $\delta_{mass} = 10$ PPM for QTOF data, $5$ PPM
    for FTMS data. We merged all features matching the same composition. Then, for
    each adduct combination, we searched the glycan database for compositions
    whose neutral mass were within $\delta_{mass}$ of the observed neutral mass - adduct
    combination mass, followed by another round of merging chromatograms with the same
    assigned composition. We reduced the data by splitting each feature where the time
    between sequential observation was greater than $\delta_{rt} = 0.25$ minutes and
    removed features with fewer than $k = 5$ data points.%
    % The MSn filter makes comparison with MultiGlycan-ESI difficult, so this doesn't
    % make sense to include here.
    % 
    % For cases where $MS^n$ scans were
    % present, these scans were mapped onto their precursor $MS^1$ features, and evaluated for
    % glycan-like product ions, retaining only those which satisfied the signature ion
    % criterion described in section S\ref{S-sec:signature_ion_criterion}.
    We term the remaining assigned and unassigned chromatograms \textit{candidate features}.
