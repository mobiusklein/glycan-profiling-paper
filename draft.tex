\documentclass{article}
\usepackage{amsmath}
\usepackage[margin=0.5in]{geometry}
\usepackage{breakcites}
\usepackage{natbib}
\usepackage{mathtools}
\DeclarePairedDelimiter\ceil{\lceil}{\rceil}
\DeclarePairedDelimiter\floor{\lfloor}{\rfloor}

\begin{document}
\newcommand{\monosaccharide}[1]{{\bf #1}}

\title{Application of Network Smoothing to Glycan LC-MS Profiling}
\author{Joshua Klein}
\begin{abstract}
    \textbf{Motivation:} Glycosylation is one of the most heterogenous
    and complex post-translational modifications, but.\\
    \textbf{Results:} These are the resutls for this article.\\
\end{abstract}

\maketitle

\section{Introduction}
Glycosylation is one of the most pervasive forms of post-translational
modification.


\section{Methods}

\subsection{Glycan Hypothesis Generation}
N-glycans start with a common core, but growing outwards, they can become quite
complex \cite{Stanley2009}. Starting with the core motif of {\bf HexNAc2 Hex3}, all
combinations of monosaccharides ranging between:$$
\begin{tabular}{c c c}
    Monosaccharide & Lower Limit & Upper Limit \\
    \monosaccharide{HexNAc} & 2 & 9 \\
    \monosaccharide{Hex} & 3 & 10 \\
    \monosaccharide{Fuc} & 0 & 4 \\
    \monosaccharide{NeuAc} & 0 & 5 \\
\end{tabular}
$$subject to to the limitation $\monosaccharide{HexNAc} > \monosaccharide{Fuc}$ and $(\monosaccharide{HexNAc} - 1) > 
\monosaccharide{NeuAc}$. We created a copy of this database for native, reduced and permethylated,
and deuteroreduced and permethylated.

\subsection{LC-MS Preprocessing}
Raw LC-MS data must be preprocessed in a number of ways before it can readily be
compared to theoretical glycan compositions. We applied a background reduction
method based upon \cite{Kaur2006}, using a window length of 2 m/z. Next, we picked
peaks using a simple gaussian model. Scans were then subjected to iterative charge
state deconvolution and deisotoping using an averagine \cite{Senko1995} formula
appropriate to the molecule under study. For native glycans, the formula was
{\bf H 1.690 C 1.0 O 0.738 N 0.071}, for permethylated glycans, the formula was
{\bf H 1.819 C 1.0 O 0.431 N 0.042}. We used an iterative approach which combines
aspects of the dependence graph method \cite{Liu2010} and with subtraction. 

\subsection{LC-MS Feature Aggregation}
We aggregated deconvoluted peaks over time to construct LC-MS features. We clustered
peaks whose neutral masses were within 15 parts-per-million error (PPM) of each other.
When there were multiple candidate clusters for a single peak, we used the cluster
with the lowest mass error. After all peaks were clustered, we sorted each cluster
by time, creating a list of LC-MS features.

\subsection{Glycan Composition Matching}
For each LC-MS feature, we queried the target glycan database for compositions whose
masses were within $\delta_{mass} = 10$ PPM mass error for QTOF data, $5$ PPM mass error
for Orbitrap data. We merged all features matching the same composition. Then, for
each adduct combination, we searched the target glycan database for compositions
whose neutral mass were within $\delta_{mass}$ of the observed neutral mass - adduct
combination mass, followed by another round of merging LC-MS features with the same
assigned composition. We reduced the data by splitting each feature where the time
between sequential observation was greater than $\delta_{rt} = 0.25$ minutes and
removed features with fewer than $k = 5$ data points. We termed the remaining assigned
and unassigned LC-MS features {\em candidate features}.

% This section runs quite long but is necessary to define the "Observed Score"
% which the network smoothing method is supposed to smooth over. I doubt I'll be
% able to include all of this section, the definition of the steps involved in
% defining the network, and the network smoothing itself.
\subsection{Feature Evaluation}
For each candidate feature, we computed several statistics to estimate how distinguishable
the observed signal was from random noise. We use the following quantities from each LC-MS
feature:

\begin{center}
\begin{tabular}{l | p{9cm}}
    \hline
    $\text{neutral mass}_i$ & The neutral mass of the $i$th chromatogram\\
    $\text{intensity}_i$ & The total intensity array assigned to the $i$th chromatogram\\
    $\text{node intensity}_{i, j}$ & The sum of all peak intensities for peaks observed in
                                     the $j$th scan for the $i$th chromatogram\\
    $\text{intensity}_{i, \text{charge}=j}$ & The total intensity assigned to the $i$th
                                              chromatogram with charge state $j$\\
    $\text{time}_{i, j}$ & The time of the $j$th scan of the $i$th chromatogram\\
    $\text{charges}_i$ & The set of charge states observed for the $i$th chromatogram\\
    $\text{peak}_{i, j}$ & The $j$th deconvoluted MS peak assigned to the $i$th chromatogram\\
    $\text{peak intensity}_{i, j}$ & The intensity assigned to $\text{peak}_{i, j}$\\
    $\text{envelope}_{i, j}$ & The normalized experimental isotopic envelope composing
                               $\text{peak}_{i, j}$, a sequence of size $K$, whose members
                               sum to $1$\\
    $\text{adducts}_i$ & The set of adduction states observed for the $i$th chromatogram\\
    $\text{intensity}_{i, \text{adduct}=j}$ & The total intensity assigned to the $i$th
                                              chromatogram with adduct $j$\\
\end{tabular}
\end{center}

\subsubsection{Chromatographic Peak Shape}
An LC-MS elution profile should be composed of one or more peak-like components, each
following a bi-Gaussian peak shape model \cite{Yu2010} or in less ideal chromatographic
circumstances, a skewed Gaussian peak shape model. We fit these models using non-linear
least squares (NLS). As measures of goodness of fit are not generally available for NLS,
we used the following criterion:\begin{align}
    {\hat y_i} &= NLS(\text{node intensity}_i, \text{time}_i) \\
    e_{i, NLS} &= \text{node intensity}_i - {\hat y_i} \\
    {\bar y_i} &= \frac{1}{n}\sum_j^n(\text{node intensity}_{i, j}) \\
    e_{i, null} &= \text{node intensity}_i - {\bar y_i} \\
    \text{line score}_i &= 1 - \frac{\sum{e_{i, NLS}^2}}{\sum{e_{i, null}^2}}
\end{align} where line score describes how much the peak shape fit improves on a flat line
fit null model.

We applied two competitive peak fitting strategies to address distorted, overlapping, or
multimodal elution profiles. The first worked iteratively by finding a best-matching peak
shape using non-linear least squares, subtracting the fitted signal and checked if there was
another peak with at least half as tall as the removed peak, if so repeating the process until
no peak can be found, saving each peak model so constructed. The second approach started
by locating local minima between putative peaks, and partitioning the LC-MS feature into
sub-groups which would be fitted independently. This method generates a candidate list of
minima, and selects the case which has the greatest difference between the minimum and its
pair of maxima to split the feature at. The strategy which produced the maximum {\em line score}
was chosen.

\subsubsection{Composition Dependent Charge State Distribution}
% Does this principle need a citation?
As the number of monosaccharides composing a glycan increases, the number of possible sites
for charge localization increases. Under normal conditions, we would expect to observe the
same molecule in multiple charge states \cite{Maxwell2012}. Which charge states are
expected would depend upon the size of the molecule and it's constituent units'
electronegativity. In it's native state, \monosaccharide{NeuAc}'s acidic group causes
glycans with one or more \monosaccharide{NeuAc} to have a propensity for higher negative
charge states\cite{Varki2009}. To capture this relationship, we modeled the probability of
observing a glycan composition for sialylated and unsialylated compositions separately.
\begin{align}
    m_i &= (\floor*{(\text{neutral mass}_i / w) / 10} + 1) * 10 \\
    \text{charged intensity}_{i,j} &= \frac{
        \text{intensity}_{i, \text{charge}=j}}{\text{intensity}_i} \\
    P(c, m) &= |m|\sum_{m_i \in m} \text{charged intensity}_{i, j} \\
    \text{charge score}_i &= \sum_{c_{i, j} \in \text{charges}_i}{P(c_{i, j}, m_i)} \\
\end{align} where $w$ is the width of the mass bin divided by 10 and $P(c, m)$ is defined as
part of the model estimation procedure.

\subsubsection{Isotopic Pattern Consistency}
Our ahead-of-time deconvolution procedure uses an averagine isotopic model and does not
capture the consistency of the isotopic pattern that was fit with the isotopic pattern
of the glycan composition that matched that peak. The criterion \begin{align}
    \text{isotope score}_i &= 1 - \frac{2}{\text{intensity}_i}\sum_j^J{
        \text{peak intensity}_{i, j}\sum_k^K{
            |\text{envelope}_{i, j, k}(
                \ln(\text{envelope}_{i, j, k}) - \ln(\text{tid}_{i, k})
                )|
        }
    }
\end{align} where {\em tid} is the theoretical isotopic pattern derived from either the $i$th
glycan composition or an averagine interpolated for $\text{neutral mass}_i$. This computes an
peak intensity weighted mean G-test comparing the goodness of fit between the experimental
envelope and the theoretical isotopic pattern.

\subsubsection{Observation Spacing Score}
The less time between observations of a glycan composition the less likely the LC-MS feature
is to contain peaks missing or caused by isotopic pattern interference or missing information.
\begin{align}
    \text{spacing score}_i &= 1 - \frac{2}{\text{intensity}_i}\sum_{j=1}^J\text{node intensity}_{i, j}(
        \text{time}_{i, j} - \text{observed time}_{i, j - 1})
\end{align}

\bibliographystyle{natbib}
\bibliography{bibliography}

\end{document}