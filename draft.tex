\documentclass{article}
\usepackage{amsmath}
\usepackage[margin=0.5in]{geometry}
\usepackage{breakcites}
\usepackage{natbib}

\begin{document}

\title{Application of Network Smoothing to Glycan LC-MS Profiling}
\author{Joshua Klein}
\begin{abstract}
    \textbf{Motivation:} Glycosylation is one of the most heterogenous
    and complex post-translational modifications, but .\\
    \textbf{Results:} These are the resutls for this article.\\
\end{abstract}

\maketitle

\section{Introduction}
Glycosylation is one of the most pervasive forms of post-translational
modification.


\section{Methods}

\subsection{Glycan Hypothesis Generation}
N-glycans start with a common core, but growing outwards, they can become quite
complex \cite{Stanley2009}. Starting with the core motif of {\bf HexNAc2 Hex3}, all
combinations of monosaccharides ranging between:$$
\begin{tabular}{c c c}
    Monosaccharide & Lower Limit & Upper Limit \\
    {\bf HexNAc} & 2 & 9 \\
    {\bf Hex} & 3 & 10 \\
    {\bf Fuc} & 0 & 4 \\
    {\bf NeuAc} & 0 & 5 \\
\end{tabular}
$$subject to to the limitation ${\bf HexNAc} > {\bf Fuc}$ and $({\bf HexNAc} - 1) > 
{\bf NeuAc}$. We created a copy of this database for native, reduced and permethylated,
and deuteroreduced and permethylated.

\subsection{LC-MS Preprocessing}
Raw LC-MS data must be preprocessed in a number of ways before it can readily be
compared to theoretical glycan compositions. We applied a background reduction
method based upon \cite{Kaur2006}, using a window length of 2 m/z. Next, we picked
peaks using a simple gaussian model. Scans were then subjected to iterative charge
state deconvolution and deisotoping using an averagine \cite{Senko1995} formula
appropriate to the molecule under study. For native glycans, the formula was
{\bf H 1.690 C 1.0 O 0.738 N 0.071}, for permethylated glycans, the formula was
{\bf H 1.819 C 1.0 O 0.431 N 0.042}. We used an iterative approach which combines
aspects of the dependence graph method \cite{Liu2010} and with subtraction. 

\subsection{LC-MS Feature Aggregation}
We aggregated deconvoluted peaks over time to construct LC-MS features. We clustered
peaks whose neutral masses were within 15 parts-per-million error (PPM) of each other.
When there were multiple candidate clusters for a single peak, we used the cluster
with the lowest mass error. After all peaks were clustered, we sorted each cluster
by time, creating a list of LC-MS features.

\subsection{Glycan Composition Matching}
For each LC-MS feature, we queried the target glycan database for compositions whose
masses were within $\delta_{mass} = 10$ PPM mass error for QTOF data, $5$ PPM mass error
for Orbitrap data. We merged all features matching the same composition. Then, for
each adduct combination, we searched the target glycan database for compositions
whose neutral mass were within $\delta_{mass}$ of the observed neutral mass - adduct
combination mass, followed by another round of merging LC-MS features with the same
assigned composition. We reduced the data by splitting each feature where the time
between sequential observation was greater than $\delta_{rt} = 0.25$ minutes and
removed features with fewer than $k = 5$ data points. We termed the remaining assigned
and unassigned LC-MS features {\em candidate features}.

\subsection{Feature Evaluation}
For each candidate feature, we computed several statistics to estimate how distinguishable
the observed signal was from random noise.

\subsubsection{Chromatographic Peak Shape}
An LC-MS elution profile should be composed of one or more peak-like components, each
following a bi-Gaussian peak shape model \cite{Yu2010} or in less ideal chromatographic
circumstances, a skewed Gaussian peak shape model. We fit these models using non-linear
least squares (NLS). As measures of goodness of fit are not generally available for NLS,
we used the following criterion:\begin{align}
    {\hat y_i} &= NLS(intensity_i, time_i) \\
    e_{i, NLS} &= intensity_i - {\hat y_i} \\
    {\bar y_i} &= \frac{1}{|intensity_i|}\sum(intensity_i) \\
    e_{i, null} &= intensity_i - {\bar y_i} \\
    \text{line score}_i &= 1 - \frac{\sum{e_{i, NLS}^2}}{\sum{e_{i, null}^2}}
\end{align}
We applied two competitive peak fitting strategies to address distorted, overlapping, or
multimodal elution profiles. The first worked iteratively by finding a best-matching peak
shape using non-linear least squares, subtracting the fitted signal and checked if there was
another peak with at least half as tall as the removed peak, if so repeating the process until
no peak can be found, saving each peak model so constructed. The second approach started
by locating local minima between putative peaks, and partitioning the LC-MS feature into
sub-groups which would be fitted independently. This method generates a candidate list of
minima, and selects the case which has the greatest difference between the minimum and its
pair of maxima to split the feature at.

\subsubsection{Composition Dependent Charge State Distribution}


\bibliographystyle{natbib}
\bibliography{bibliography}

\end{document}