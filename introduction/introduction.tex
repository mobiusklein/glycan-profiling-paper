\section{Introduction}
Glycosylation is one of the most pervasive and diverse forms of
post-translational modification (\cite{Varki2017}). Their study
is of great importance for understanding broad classes of biological
processes. Mass spectrometry (MS) is a powerful tool for glycan
analysis (\cite{Zaia2008}). While unseparated MS experiments using
methods like MALDI provide strong signal, but cannot interpret complex
mixtures \cite{Peltoniemi2013}. An online separation method like liquid
chromatography (LC) or capillary electrophoresis (CE) makes analyzing
such complex samples possible, at the cost of increased analytical
complexity.

There are many tools for interpreting glycan mass spectral datasets (
\cite{Yu2013,Peltoniemi2013,Kronewitter2014,Goldberg2009,Maxwell2012,Ceroni2008,Frank2010})
for both unseparated and separated experimental protocols. The different types
of instrumentation these programs were written to accommodate introduces
different types of signal processing approaches, for example SysBioWare
(\cite{Frank2010}) performs sophisticated baseline removal prior to fitting
peaks, while tools like GlyQ-IQ (\cite{Kronewitter2014}) was written for much
cleaner Fourier Transform MS (FTMS), and so does not accommodate these well.
Tools that build on the THRASH implementation from Decon2LS (\cite{Jaitly2009,Yu2013,Maxwell2012})
are likewise unable to deal with variable baseline noise or extreme dynamic range.

Each tool also has its own format for defining glycan structures or compositions,
some even bundling a large database with their software to remove the burden from
the user to build a list of candidates themselves (\cite{Yu2013,Kronewitter2014,Goldberg2009})
while others define methods for building glycan databases as part of the program (
\cite{Maxwell2012,Ceroni2008}). Many of these tools are designed for specific glycan
subclass such as N-glycans or glycosaminoglycans, limiting their vocabulary of
possible monosaccharides to just those found in that subclass (\cite{Yu2013,Kronewitter2014,Peltoniemi2013,Goldberg2009}),
Often, these tools are tailored to analysis of a particular derivatization state,
adduction conditions, or neutral loss pattern (\cite{Yu2013,Peltoniemi2013,Maxwell2012}).

These spectral processing and glycan library properties are reflected in the
scoring function that each program uses to discriminate glycan signal from
the background noise and contaminants. As observed by \cite{Goldberg2009}, there
is value in including related glycan composition identifications in how much
confidence one assigns to a another glycan composition assignment. They use a method
to exploit the known biosynthetic rules to connect peaks in a MALDI spectrum which
could be assigned to a particular \nglycan by intact mass alone. Their method using
the maximum weighted subgraph of the biosynthetic network in one of their three  had
demonstrably better performance than chance with their expert system annotation method.
Kronewitter and colleagues considered a similar idea with more emphasis on handling
in-source fragmentation (\cite{Kronewitter2014}) observed in LC-MS and LC-MS/MS
experiments.

We extend this notion of a glycan family to cover more sectors of the biosynthetic
landscape which we term "neighborhoods", and present an algorithm for learning the
importance of each neighborhood from observed data, which can in turn be used to
improve glycan composition assignment performance.