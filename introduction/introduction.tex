\section{Introduction}

Glycosylation modulates the structures and functions of proteins and lipids
in a broad class of biological processes (\cite{Varki2017}). Accurate mass
measurement defines monosaccharide composition given assumptions regarding
glycan class and biosynthesis (\cite{Zaia2008}). For unseparated mixtures,
mass spectrometry analysis determines the mass-to-charge ratio values for
only the most abundant glycans; dynamic range for detection of glycans is
poor because of ion suppression (\cite{Peltoniemi2013}). By contrast, online
separations coupled with mass spectrometry improve dynamic range and
reproducibility of glycan analysis, at the cost of increased analysis time
and workflow complexity.

There are many tools for interpreting glycan mass spectral datasets (
\cite{Yu2013,Peltoniemi2013,Kronewitter2014,Goldberg2009,Maxwell2012,Ceroni2008,Frank2010})
for both unseparated and separated experimental protocols. These programs address
instrument-specific signal processing requirements. For example SysBioWare
(\cite{Frank2010}) performs sophisticated baseline removal prior to fitting
peaks, while GlyQ-IQ (\cite{Kronewitter2014}) was written for cleaner Fourier Transform
MS (FTMS) that does not require such a baseline removal step. Tools that build on
the THRASH implementation from Decon2LS (\cite{Jaitly2009,Yu2013,Maxwell2012})
are unable to deal with variable baseline noise or extreme dynamic range.

Each tool also has its own format for defining glycan structures or compositions,
some even bundling a large database with their software to remove the burden from
the user to build a list of candidates themselves (\cite{Yu2013,Kronewitter2014,Goldberg2009})
while others define methods for building glycan databases as part of the program (
\cite{Maxwell2012,Ceroni2008}). Many of these tools are designed for specific glycan
subclass such as \nglycans or glycosaminoglycans and/or organisms, limiting their
vocabulary of possible monosaccharides to just those commonly found in that subgroup
(\cite{Yu2013,Kronewitter2014,Peltoniemi2013,Goldberg2009}). Often, these tools are
tailored for analysis of a particular derivatization state, adduction conditions, or
neutral loss pattern (\cite{Yu2013,Peltoniemi2013,Maxwell2012}). Work has been done
to construct a standardized namespace and representation for glycans, \glyspace, including both
structures and compositions (\cite{Tiemeyer2017,Campbell2014}). This data is publicly
accessible, including a programmatic query interface using SPARQL over HTTPS (
\cite{Aoki-Kinoshita2015}). Tools that can communicate with these services have the
potential to lead researchers to find deeper connections from cross-referenced information,
and other researchers can more readily find and use their work.

These spectral processing and glycan library properties are reflected in the
scoring function that each program uses to discriminate glycan signal from
the background noise and contaminants. Several methods have been developed
using different facets of the observed data. \cite{Yu2013} used the isotopic pattern
goodness-of-fit while \cite{Peltoniemi2013} used intensity features of associated \msn
scans to evaluate partial structure and composition match quality. \cite{Kronewitter2014} combined
several features of the $MS^1$ evidence, including elution profile peak shape goodness-of-fit,
isotopic fit, mass accuracy, scan count, and in-source fragmentation correlation. Some of these methods
are well-defined and invariant from instrument to instrument in this era of high resolution
mass spectrometry, but others are tightly coupled to the experimental equipment. Missing from
this list are methods to target a glycan's intrinsic properties, such as charge state distribution
or facility in acquiring adducts, which can increase the number of spurious assignments if not
considered. We propose a new scoring function which is able to combine those properties which
are independent of experimental setup with these glycan-aware features.

As observed by \cite{Goldberg2009}, there is also value in including related glycan
composition identifications in how much confidence one assigns to a given glycan
composition assignment. They used a method to exploit the known biosynthetic rules of
\nglycans to connect peaks in a MALDI spectrum assigned to a particular \nglycan by
intact mass alone. Their method using the maximum weighted subgraph of the biosynthetic
network had demonstrably better performance than chance with their expert system annotation
method. \cite{Kronewitter2014} considered a similar idea with more emphasis on handling
in-source fragmentation observed in LC-MS and LC-MS/MS experiments.

We extend this notion of a glycan family to cover more sectors of the biosynthetic
landscape which we term ``neighborhoods'', and present an algorithm for learning the
importance of each neighborhood from observed data, which can in turn be used to
improve glycan composition assignment performance. We also apply our method using
three different glycan composition search spaces to show how the underlying database
can influence results. We present our method on typical \nglycans in humans, though our
method can be applied to any variety of glycan composition whose monosaccharides can be
described using IUPAC trivial names or or whose components can be described in terms of
chemical formulae.
