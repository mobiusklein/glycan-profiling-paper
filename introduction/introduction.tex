\section{Introduction}
Glycosylation is one of the most pervasive and diverse forms of
post-translational modification (\cite{Varki2017}). Their study
is of great importance for understanding broad classes of biological
processes. Mass spectrometry (MS) is a powerful tool for glycan
analysis (\cite{Zaia2008}). MS of unseparated mixtures results in poor
dynamic range for detection of glycans \cite{Peltoniemi2013}. An online
separation method like liquid chromatography (LC) or capillary
electrophoresis (CE) improves dynamic range and reproducibility of
glycan detection, at the cost of increased analysis time and workflow
complexity.

There are many tools for interpreting glycan mass spectral datasets (
\cite{Yu2013,Peltoniemi2013,Kronewitter2014,Goldberg2009,Maxwell2012,Ceroni2008,Frank2010})
for both unseparated and separated experimental protocols. These programs address
instrument-specific signal processing requirements. For example SysBioWare
(\cite{Frank2010}) performs sophisticated baseline removal prior to fitting
peaks, while GlyQ-IQ (\cite{Kronewitter2014}) was written cleaner Fourier Transform
MS (FTMS) that does not require such a baseline removal step. Tools that build on
the THRASH implementation from Decon2LS (\cite{Jaitly2009,Yu2013,Maxwell2012})
are unable to deal with variable baseline noise or extreme dynamic range.

Each tool also has its own format for defining glycan structures or compositions,
some even bundling a large database with their software to remove the burden from
the user to build a list of candidates themselves (\cite{Yu2013,Kronewitter2014,Goldberg2009})
while others define methods for building glycan databases as part of the program (
\cite{Maxwell2012,Ceroni2008}). Many of these tools are designed for specific glycan
subclass such as \nglycans or glycosaminoglycans, limiting their vocabulary of
possible monosaccharides to just those commonly found in that subclass
(\cite{Yu2013,Kronewitter2014,Peltoniemi2013,Goldberg2009}). Often, these tools are
tailored for analysis of a particular derivatization state, adduction conditions, or
neutral loss pattern (\cite{Yu2013,Peltoniemi2013,Maxwell2012}). Work has been done
to construct a standardized namespace and representation for glycans, including both
structures and compositions (\cite{Tiemeyer2017,Campbell2014}). This data is publicly
accessible, including a programmatic query interface using SPARQL over HTTPS (
\cite{Aoki-Kinoshita2015}). Tools that can communicate with these services have the
potential to lead researchers to find deeper connections from cross-referenced information,
and other researchers can more readily find and use their work.

These spectral processing and glycan library properties are reflected in the
scoring function that each program uses to discriminate glycan signal from
the background noise and contaminants. Several methods have been developed
using different facets of the observed data. \cite{Yu2013} used the isotopic pattern
goodness-of-fit while \cite{Peltoniemi2013} used intensity features
of associated \msn scans to. \cite{Kronewitter2014} combined several features of the
$MS^1$ evidence, including elution profile peak shape goodness-of-fit, isotopic fit,
mass accuracy, scan count, and in-source fragmentation correlation. Some of these methods
are well-defined and invariant from instrument to instrument in this era of high resolution
mass spectrometry, but others are tightly coupled to the experimental equipment. Missing from
this list are methods to target a glycan's intrinsic properties, such as charge state distribution
or facility in acquiring adducts, which can increase the number of spurious assignments when not
accounted for. We propose a new scoring function which is able to combine those properties which
are independent of experimental setup with these glycan-aware features.

As observed by \cite{Goldberg2009}, there is also value in including related glycan
composition identifications in how much confidence one assigns to a another glycan
composition assignment. They use a method to exploit the known biosynthetic rules of
\nglycans to connect peaks in a MALDI spectrum which could be assigned to a particular \nglycan by
intact mass alone. Their method using the maximum weighted subgraph of the biosynthetic
network in one of their three had demonstrably better performance than chance with
their expert system annotation method. \cite{Kronewitter2014} considered a similar idea with
more emphasis on handling in-source fragmentation observed in LC-MS and LC-MS/MS experiments.

We extend this notion of a glycan family to cover more sectors of the biosynthetic
landscape which we term "neighborhoods", and present an algorithm for learning the
importance of each neighborhood from observed data, which can in turn be used to
improve glycan composition assignment performance. We also apply our method using
three different glycan composition search spaces to show how the underlying database
can influence results.
