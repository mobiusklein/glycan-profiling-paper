
\section{glySpace Integration and Upload}\label{sec:glyspace_integration_and_upload}

    We extracted \nglycan structures from GlyTouCan Query Endpoint (http://ts.glytoucan.org/sparql) using the SPARQL query

    \begin{verbatim}
    PREFIX glycan: <http://purl.jp/bio/12/glyco/glycan#>
    PREFIX skos: <http://www.w3.org/2004/02/skos/core#>
    PREFIX glycoinfo: <http://rdf.glycoinfo.org/glycan/>

    SELECT DISTINCT ?saccharide ?glycoct ?motif WHERE {
        ?saccharide a glycan:saccharide .
        ?saccharide glycan:has_glycosequence ?sequence .
        ?saccharide skos:exactMatch ?gdb .
        ?gdb glycan:has_reference ?ref .
        ?ref glycan:is_from_source ?source .
        ?source glycan:has_taxon ?taxon
        FILTER CONTAINS(str(?sequence), "glycoct") .
        ?sequence glycan:has_sequence ?glycoct .
        ?saccharide glycan:has_motif ?motif .
        FILTER(?motif in (glycoinfo:G00026MO))
    }
    \end{verbatim} and converted each structure into a glycan composition, followed
    by substituent separation for sulfated and phosphorylated monosaccharides, and
    filtering out compositions containing units not in \textbf{[Hex, HexNAc, Fuc, Neu5Ac, sulfate]}.
    This procedure is implemented in Python in the included ``glyspace\_extract\_nglycans.py'' script.

    Note that lines 4-7 restricts the query to only compositions which were in Glycome-DB which came
    from externally curated sources with taxonomic information, though it is not limited to human \nglycans
    specifically. If these lines are omitted, the query will return over 800 compositions, compared to
    the expected 275, but the additional compositions will not have been curated. The precise number of
    compositions returned by this modified query is not fixed as GlyTouCan is a living database, accepting new
    submissions.

    We converted our \nglycan compositions into partially determined topologies
    assuming that the chitobios core was present to ensure that they were classified
    as \nglycans.

    From \rpserum
    \begin{verbatim}
        {Fuc:1; Hex:5; HexNAc:3; Neu5Ac:1}
        {Fuc:2; Hex:5; HexNAc:4; Neu5Ac:2}
        {Fuc:2; Hex:6; HexNAc:5; Neu5Ac:3}
        {Fuc:2; Hex:7; HexNAc:6; Neu5Ac:3}
        {Fuc:2; Hex:7; HexNAc:6; Neu5Ac:4}
        {Hex:7; HexNAc:6; Neu5Ac:2}
        {Hex:7; HexNAc:6; Neu5Ac:3}
        {Hex:8; HexNAc:7; Neu5Ac:3}
        {Hex:8; HexNAc:7; Neu5Ac:4}
        {Hex:9; HexNAc:8; Neu5Ac:2}
    \end{verbatim}

    From \philbs
    \begin{verbatim}
        {@sulfate:1; Fuc:1; Hex:4; HexNAc:5}
        {@sulfate:1; Fuc:1; Hex:5; HexNAc:4}
        {@sulfate:1; Fuc:1; Hex:5; HexNAc:5}
        {@sulfate:1; Fuc:2; Hex:4; HexNAc:5}
        {@sulfate:1; Fuc:2; Hex:6; HexNAc:5}
        {@sulfate:1; Fuc:2; Hex:9; HexNAc:8}
        {@sulfate:1; Fuc:3; Hex:4; HexNAc:5}
        {@sulfate:1; Fuc:3; Hex:6; HexNAc:5}
        {@sulfate:1; Fuc:3; Hex:9; HexNAc:8}
        {@sulfate:1; Fuc:4; Hex:6; HexNAc:5}
        {@sulfate:1; Fuc:4; Hex:8; HexNAc:7}
        {@sulfate:1; Fuc:4; Hex:9; HexNAc:8}
        {@sulfate:1; Hex:10; HexNAc:9}
        {@sulfate:1; Hex:4; HexNAc:5}
        {@sulfate:1; Hex:5; HexNAc:4}
        {Fuc:2; Hex:8; HexNAc:7}
        {Fuc:3; Hex:7; HexNAc:6}
        {Fuc:3; Hex:8; HexNAc:7}
        {Fuc:4; Hex:8; HexNAc:7}
        {Hex:10; HexNAc:9}
    \end{verbatim}