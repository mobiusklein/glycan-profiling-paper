\section{Discussion}

    We demonstrate that this regularization method can improve the
    sensitivity and specificity of glycan composition assignment for
    LC-MS based experiments. The method we proposed uses similar
    assumptions about the importance of common substructural elements
    of \nglycans to \cite{Goldberg2009}, but we extend this concept
    with the addition of a procedure for learning the relationship
    strengths and use broader groups of structures.

    The experimental results from the original analysis of \philbs and
    \phil82 demonstrated that while both strains expressed predominantly
    high-mannose glycosylation, \philbs expressed more larger complex-type
    structures (\cite{Khatri2016a}). In our findings, we recapitulate
    these results while reducing the number of false positive assignments.
    There are substantial differences in both the mass spectral processing
    and scoring schemes which contribute to these results, but the regularization
    procedure is responsible for recovering many low abundance features from
    this comparison. As these samples are derived from an avian tissue, we
    may be able to observe larger branching patterns than are observed in
    normal mammalian tissue (\cite{Stanley2009}). There is evidence for this
    in the \philbs with HexNAc9Hex10-based compositions suggesting a seven
    branch pattern, though this cannot be determined without high quality
    \msn. The $\mathbf{\tau}$ fit for both strains have smaller values
    in the neighborhoods of their largest glycan compositions as these
    features tended to be low in abundance and not high scoring in their
    own right, but were partially supported by the overlap with the next
    largest neighborhood, as expected.

    We reproduce the majority of the glycan assignments from \cite{Yu2013}.
    however the ambiguity caused by ammonium adduction as shown in
    Figure~\ref{fig:rpserum_assignment:c} makes a direct comparison of
    composition assignment lists difficult. Out of the eight missed compositions,
    four were missing because of insufficient data points to fit a peak shape,
    which requires at least five points. The other four were not detected either
    due to mass error, \cite{Yu2013} used 10 ppm while we used 5 ppm for FT-MS data,
    or due to low level signal processing decisions. Since we were unable to
    reproduce the published results from \cite{Yu2013} using their software and
    accompanying, it was not reasonable to adapt our composition database to work
    with their software and run a side-by-side test to demonstrate how many
    additional glycan compositions one algorithm identifies compared to another
    without bias.

    Of the compositions assigned by this algorithm that were not
    mentioned in \cite{Yu2013} but were annotated in the original publication
    of this dataset in \cite{Hu2012} include \textbf{HexNAc3 Hex4},
    \textbf{HexNAc3 Hex4 NeuAc}, and \textbf{HexNAc5 Hex3}. Because our database
    was constructed based on combinatorial rules that did not take into account
    all biosynthetic constraints, we include infeasible compositions in our search
    space, such as \textbf{HexNAc2 Hex10 Fuc} and \textbf{HexNAc5 Hex3 Fuc1 NeuAc2}.
    Future work could be done to restrict the database to only biosynthetically
    feasible glycan compositions. This would also have benefits for the construction
    of the composition network where only those compositions which have an enzymatic
    reaction to from one to the other would have an edge connecting them, such that
    \textbf{HexNAc5 Hex6 NeuAc2} would not have an edge to \textbf{HexNAc5 Hex7 NeuAc2}
    as in our current model.
    
    These invalid glycan compositions can match LC-MS features at any point in
    the elution profile, though in this dataset the majority of these matches
    appear to match in the time range between 10 and 22 minutes, and similar glycan
    compositions that are biosynthetically valid elute later on in the experiment.
    This indicates a need for a retention-time aware approach to evaluating glycan
    composition assignments, as described in \cite{Hu2016}, but this is likely dependent
    upon the experimental workup and separation technique used. The definition of our
    composition graph and its neighborhoods also mitigates this to some extent, though
    it depends upon the feature scoring metrics to determine whether a feature is
    eligble for smoothing. These metrics were based upon a limited sampling of data
    and could be improved by acquiring more training sets to build more granular
    models in the case of the charge state and adduct scores.

\section{Conclusions}
    In this study, we demonstrated the advantages of our application of Laplacian
    Regularization to smooth LC-MS assignments of glycan compositions across multiple
    experimental protocols (\cite{Hu2012, Khatri2016a}). Our algorithm's performance is
    competative with existing tools.

    All of the methods demonstrated in this paper are available as part of the open source,
    cross-platform glycomics and glycoproteomics software \texttt{GlycReSoft}, freely
    available at \href{http://www.bumc.bu.edu/msr/glycresoft/}{http://www.bumc.bu.edu/msr/glycresoft/}.
