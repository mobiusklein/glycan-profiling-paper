\section{Discussion}

    We demonstrated that the regularization method improved the
    sensitivity and specificity of glycan composition assignment for
    LC-MS based experiments. The method used similar assumptions about
    the importance of common substructural elements of \nglycans to
    \cite{Goldberg2009}, but we extend this concept with the addition
    of a procedure for learning the relationship strengths and use
    broader groups of structures.

    The experimental results from the original analysis of \philbs and
    \phil82 demonstrated that while both strains expressed predominantly
    high-mannose glycosylation, \philbs expressed more larger complex-type
    structures (\cite{Khatri2016a}). In our findings shown in Figure~\ref{fig:philbs_assignments},
    we recapitulate these results while reducing the number of false
    assignments, Table~\ref{tab:philbs_statistics}. There are substantial
    differences in both the mass spectral processing and scoring schemes which
    contribute to these results, but the regularization procedure is responsible
    for recovering many low abundance features from this comparison. As these
    samples are derived from chicken eggs, we have observed larger
    branching patterns than are observed in normal mammalian tissue (\cite{Stanley2009}).
    There is evidence for this in the \philbs with \textbf{HexNAc9 Hex10}-based compositions
    suggesting a seven branch pattern, though this cannot be determined without high quality
    \msn data. The $\mathbf{\tau}$ fit for Phil-BS (shown) and Phil-82 (supplement) have
    smaller values in the neighborhoods of their largest glycan compositions as
    these features tended to be low in abundance and not high scoring in their own
    right, but were partially supported by the overlap with the next largest neighborhood,
    as expected. We observed the best performance with the \textit{Combinatorial + Sulfate}
    database, which produced more than half-again as many true matches than the other two
    databases. It produced several false matches as well, but the smoothing process removed
    these while boosting the score of other low abundance matches which were consistent with
    higher scoring matches.

    The Krambeck database performed identically in all smoothing conditions as it
    was only able to match the common species, not including cases that were multiply
    fucosylated or sulfated. It had no false matches ranked alongside its true matches so
    smoothing could not change its performance. The \glyspace-derived database produced more true
    matches, but also lacked some of these more fucosylated and complex compositions. Some of the
    compositions included by the \glyspace-derived database were lower scoring, but the chosen
    value of $\gamma$ for that database was greater than 18, causing the fitted values of $\mathbf{\tau}$
    omit the larger, less abundant complex-type \nglycans. This caused smoothing to lower the scores
    of these real matches rather than raise them, as with the \textit{Combinatorial + Sulfate} database.

    As we show in Figure~\ref{fig:rpserum_perf}, regularization improves the
    predictive performance of the identification algorithm on \rpserum for all databases.
    We reproduce the majority of the glycan assignments from \cite{Yu2013}, but the ambiguity
    caused by ammonium adduction as shown in Figure~\ref{fig:rpserum_assignment:c} makes a
    direct comparison of composition assignment lists difficult. Our algorithm requires a minimum
    amount of MS1 information in order to compute a score, which some of the assignments in the
    original published results do not possess, which are omitted from the count in Table~
    \ref{tab:serum_statistics}. After accounting for ambiguity, we were able to assign all
    of the compositions previously reported using the Krambeck database, which was used
    by \cite{Yu2013}, and with the combinatorial database. The \glyspace-derived database did not
    contain all of these compositions, but performed competitively with the combinatorial
    database's ROC AUC. The combinatorial database contained a small number of glycan compositions
    which were not in Krambeck but which were consistent with other glycan compositions
    observed nearby in retention time. The combinatorial database also benefited most
    substantially from smoothing, discarding many false positives while retaining many more
    true positives at the same false positive rate compared to the other databases. These
    invalid glycan compositions can match LC-MS features at any point in the elution profile,
    though in this dataset the majority of these matches appear to be in the time range between
    10 and 22 minutes, and similar glycan compositions that are biosynthetically valid elute
    later on in the experiment. Therefore a for a retention-time aware approach to evaluating glycan
    composition assignments, as described in \cite{Hu2016} could also be useful, but this is
    likely dependent upon the experimental workup and separation technique used.

    While the biosynthetically constrained Krambeck database performed better on \rpserum,
    it did not contain all of the reasonably assignable glycan compositions, and it performed
    poorly on \philbs with a false negative rate of 50\% compared to the combinatorial database.
    This is because the necessary enzymatic pathways were either not considered in the original
    authors' model because either the enzyme was excluded for simplicity (\cite{Krambeck2009}) or
    because the particular enzymes used were not within the scope of the model used
    (\cite{Spiro2000,Ichimiya2014}). This highlights the importance of selecting a good reference
    database, though a post-processing step such as the we described here can help mitigate using
    too large a database, but not a too small one.

    In this work, we used the same network neighborhood imposed over different underlying sets of
    composition nodes, and the connectivity of those networks did not take into account the biosynthetic
    process. It may be possible to obtain better performance by defining network connectivity according
    to enzymatic relationships. This may also alter how the neighborhoods are defined and how $\mathbf{A}$
    is parameterized, and in turn how $\mathbf{\tau}$ is learned. Similarly, this procedure depends upon
    the scoring functions used, so selecting another set of functions for the data to fit may lead
    to different parameter values.

    Lastly, while these case studies have demonstrated the algorithm's ability to learn network parameters
    from the data, an expert can define $\mathbf{\tau}$ and $\mathbf{A}$ themselves or obtain a model fitted
    on related data and apply it directly without a fitting step. An expert could use this model specification
    to impose prior beliefs on the evaluation process, and adjust $\lambda$ to control the importance of
    the these beliefs. Similarly, one could also use the derivation of ${\hat \phi_m}$ to estimate the score
    for an unobserved glycan composition, given $\mathbf{A}$ and $\mathbf{\tau}$.

    We used our glycoinformatics toolkit to produce a richer abstraction of glycans and monosaccharides,
    including producing standard-compliant textual representations of these structures and compositions.
    We produced a text file containing all of the glycan compositions found in the Krambeck and Combinatorial
    database but not the \glyspace-derived database in the above samples (see supplemental information
    \ref{S-sec:glyspace_integration_and_upload}),
    and have submit it to GlyTouCan (\cite{Tiemeyer2017}) for registration so that future researchers can
    use these structures.

\section{Conclusions}
    In this study, we demonstrated the advantages of our application of Laplacian
    Regularization to smooth LC-MS assignments of glycan compositions across multiple
    experimental protocols (\cite{Hu2012, Khatri2016a}). Our algorithm's performance is
    competitive with existing tools for analyzing the same type of data, with the added
    benefit of more flexible evaluation process and broader range of understood monosaccharides.
    Our tools integrate with \glyspace and allows users to leverage existing glycomics
    repositories to build databases where applicable.

    All of the methods demonstrated in this paper are available as part of the open source,
    cross-platform glycomics and glycoproteomics software \texttt{GlycReSoft}, freely
    available at \href{http://www.bumc.bu.edu/msr/glycresoft/}{http://www.bumc.bu.edu/msr/glycresoft/}.
